
\section{Schedule} 


\parskip = 0pt


\noindent This section lists all the variables which are assigned storage by thorn Llama/LlamaWaveToy.  Storage can either last for the duration of the run ({\bf Always} means that if this thorn is activated storage will be assigned, {\bf Conditional} means that if this thorn is activated storage will be assigned for the duration of the run if some condition is met), or can be turned on for the duration of a schedule function.


\subsection*{Storage}

\hspace{5mm}

 \begin{tabular*}{160mm}{ll} 

{\bf Always:}&  ~ \\ 
 scalar[2] density[2] velocity[2] & ~\\ 
 dx\_scalar[2] & ~\\ 
 dxx\_scalar[2] & ~\\ 
 dx\_density[2] & ~\\ 
 errors & ~\\ 
 metric inverse\_metric lapse shift volume\_element & ~\\ 
 scalardot densitydot velocitydot & ~\\ 
~ & ~\\ 
\end{tabular*} 


\subsection*{Scheduled Functions}
\vspace{5mm}

\noindent {\bf CCTK\_STARTUP} 

\hspace{5mm} lwt\_startup 

\hspace{5mm}{\it register banner with cactus } 


\hspace{5mm}

 \begin{tabular*}{160mm}{cll} 
~ & Language:  & c \\ 
~ & Options:  & meta \\ 
~ & Type:  & function \\ 
\end{tabular*} 


\vspace{5mm}

\noindent {\bf MoL\_Register}   (conditional) 

\hspace{5mm} lwt\_register\_mol 

\hspace{5mm}{\it register variables with mol } 


\hspace{5mm}

 \begin{tabular*}{160mm}{cll} 
~ & Language:  & c \\ 
~ & Options:  & meta \\ 
~ & Type:  & function \\ 
\end{tabular*} 


\vspace{5mm}

\noindent {\bf CCTK\_ANALYSIS}   (conditional) 

\hspace{5mm} lwt\_calc\_rhs 

\hspace{5mm}{\it calculate the rhs } 


\hspace{5mm}

 \begin{tabular*}{160mm}{cll} 
~ & Language:  & fortran \\ 
~ & Storage:  & scalardot \\ 
~& ~ &densitydot\\ 
~& ~ &velocitydot\\ 
~ & Sync:  & scalardot \\ 
~& ~ &densitydot\\ 
~& ~ &velocitydot\\ 
~ & Triggers:  & scalardot \\ 
~& ~ &densitydot\\ 
~& ~ &velocitydot\\ 
~ & Type:  & function \\ 
\end{tabular*} 


\vspace{5mm}

\noindent {\bf CCTK\_ANALYSIS} 

\hspace{5mm} lwt\_calcenergy 

\hspace{5mm}{\it calculate the energy of the scalar field } 


\hspace{5mm}

 \begin{tabular*}{160mm}{cll} 
~ & Language:  & fortran \\ 
~ & Storage:  & scalarenergy \\ 
~ & Sync:  & scalarenergy \\ 
~ & Triggers:  & scalarenergy \\ 
~ & Type:  & function \\ 
\end{tabular*} 


\vspace{5mm}

\noindent {\bf CCTK\_ANALYSIS} 

\hspace{5mm} lwt\_error 

\hspace{5mm}{\it calculate errors of the solution } 


\hspace{5mm}

 \begin{tabular*}{160mm}{cll} 
~ & Language:  & fortran \\ 
~ & Storage:  & errorsperiodic \\ 
~ & Triggers:  & errors \\ 
~& ~ &errorsperiodic\\ 
~ & Type:  & function \\ 
\end{tabular*} 


\vspace{5mm}

\noindent {\bf CCTK\_ANALYSIS} 

\hspace{5mm} lwt\_min\_spacing 

\hspace{5mm}{\it calculate the smallest grid spacing } 


\hspace{5mm}

 \begin{tabular*}{160mm}{cll} 
~ & Language:  & fortran \\ 
~ & Storage:  & min\_spacing \\ 
~ & Sync:  & min\_spacing \\ 
~ & Triggers:  & min\_spacing \\ 
~ & Type:  & function \\ 
\end{tabular*} 


\vspace{5mm}

\noindent {\bf CCTK\_INITIAL} 

\hspace{5mm} lwt\_init\_metric 

\hspace{5mm}{\it initialise the metric } 


\hspace{5mm}

 \begin{tabular*}{160mm}{cll} 
~ & Language:  & fortran \\ 
~ & Type:  & function \\ 
\end{tabular*} 


\vspace{5mm}

\noindent {\bf CCTK\_INITIAL} 

\hspace{5mm} lwt\_calc\_inverse\_metric 

\hspace{5mm}{\it transform the metric } 


\hspace{5mm}

 \begin{tabular*}{160mm}{cll} 
~ & After:  & lwt\_init\_metric \\ 
~ & Language:  & fortran \\ 
~ & Type:  & function \\ 
\end{tabular*} 


\vspace{5mm}

\noindent {\bf CCTK\_INITIAL} 

\hspace{5mm} lwt\_init 

\hspace{5mm}{\it initialise the system } 


\hspace{5mm}

 \begin{tabular*}{160mm}{cll} 
~ & After:  & lwt\_init\_metric \\ 
~ & Language:  & fortran \\ 
~ & Type:  & function \\ 
\end{tabular*} 


\vspace{5mm}

\noindent {\bf MoL\_CalcRHS} 

\hspace{5mm} lwt\_calc\_rhs 

\hspace{5mm}{\it calculate the rhs } 


\hspace{5mm}

 \begin{tabular*}{160mm}{cll} 
~ & Language:  & fortran \\ 
~ & Type:  & function \\ 
\end{tabular*} 


\vspace{5mm}

\noindent {\bf MoL\_PostStep} 

\hspace{5mm} lwt\_outerboundary 

\hspace{5mm}{\it apply outer boundaries } 


\hspace{5mm}

 \begin{tabular*}{160mm}{cll} 
~ & Language:  & fortran \\ 
~ & Type:  & function \\ 
\end{tabular*} 


\vspace{5mm}

\noindent {\bf MoL\_RHSBoundaries} 

\hspace{5mm} lwt\_rhs\_outerboundary 

\hspace{5mm}{\it apply mol rhs outer boundaries (eg. radiative boundary condition) } 


\hspace{5mm}

 \begin{tabular*}{160mm}{cll} 
~ & Language:  & fortran \\ 
~ & Type:  & function \\ 
\end{tabular*} 


\vspace{5mm}

\noindent {\bf MoL\_PostStep} 

\hspace{5mm} lwt\_boundaries 

\hspace{5mm}{\it select the boundary condition } 


\hspace{5mm}

 \begin{tabular*}{160mm}{cll} 
~ & After:  & lwt\_outerbound \\ 
~ & Language:  & fortran \\ 
~ & Options:  & level \\ 
~ & Sync:  & scalar \\ 
~& ~ &density\\ 
~& ~ &velocity\\ 
~ & Type:  & function \\ 
\end{tabular*} 


\vspace{5mm}

\noindent {\bf MoL\_PostStep} 

\hspace{5mm} applybcs 

\hspace{5mm}{\it apply boundary conditions } 


\hspace{5mm}

 \begin{tabular*}{160mm}{cll} 
~ & After:  & lwt\_boundaries \\ 
~ & Type:  & group \\ 
\end{tabular*} 


\subsection*{Aliased Functions}

\hspace{5mm}

 \begin{tabular*}{160mm}{ll} 

{\bf Alias Name:} ~~~~~~~ & {\bf Function Name:} \\ 
ApplyBCs & LWT\_ApplyBCs \\ 
\end{tabular*} 



\vspace{5mm}\parskip = 10pt 
