% *======================================================================*
%  Cactus Thorn template for ThornGuide documentation
%  Author: Ian Kelley
%  Date: Sun Jun 02, 2002
%
%  Thorn documentation in the latex file doc/documentation.tex
%  will be included in ThornGuides built with the Cactus make system.
%  The scripts employed by the make system automatically include
%  pages about variables, parameters and scheduling parsed from the
%  relevant thorn CCL files.
%
%  This template contains guidelines which help to assure that your
%  documentation will be correctly added to ThornGuides. More
%  information is available in the Cactus UsersGuide.
%
%  Guidelines:
%   - Do not change anything before the line
%       % START CACTUS THORNGUIDE",
%     except for filling in the title, author, date, etc. fields.
%        - Each of these fields should only be on ONE line.
%        - Author names should be separated with a \\ or a comma.
%   - You can define your own macros, but they must appear after
%     the START CACTUS THORNGUIDE line, and must not redefine standard
%     latex commands.
%   - To avoid name clashes with other thorns, 'labels', 'citations',
%     'references', and 'image' names should conform to the following
%     convention:
%       ARRANGEMENT_THORN_LABEL
%     For example, an image wave.eps in the arrangement CactusWave and
%     thorn WaveToyC should be renamed to CactusWave_WaveToyC_wave.eps
%   - Graphics should only be included using the graphicx package.
%     More specifically, with the "\includegraphics" command.  Do
%     not specify any graphic file extensions in your .tex file. This
%     will allow us to create a PDF version of the ThornGuide
%     via pdflatex.
%   - References should be included with the latex "\bibitem" command.
%   - Use \begin{abstract}...\end{abstract} instead of \abstract{...}
%   - Do not use \appendix, instead include any appendices you need as
%     standard sections.
%   - For the benefit of our Perl scripts, and for future extensions,
%     please use simple latex.
%
% *======================================================================*
%
% Example of including a graphic image:
%    \begin{figure}[ht]
% 	\begin{center}
%    	   \includegraphics[width=6cm]{/home/runner/work/einsteintoolkit/einsteintoolkit/arrangements/EinsteinInitialData/Hydro_RNSID/doc/MyArrangement_MyThorn_MyFigure}
% 	\end{center}
% 	\caption{Illustration of this and that}
% 	\label{MyArrangement_MyThorn_MyLabel}
%    \end{figure}
%
% Example of using a label:
%   \label{MyArrangement_MyThorn_MyLabel}
%
% Example of a citation:
%    \cite{MyArrangement_MyThorn_Author99}
%
% Example of including a reference
%   \bibitem{MyArrangement_MyThorn_Author99}
%   {J. Author, {\em The Title of the Book, Journal, or periodical}, 1 (1999),
%   1--16. {\tt http://www.nowhere.com/}}
%
% *======================================================================*

\documentclass{article}

% Use the Cactus ThornGuide style file
% (Automatically used from Cactus distribution, if you have a
%  thorn without the Cactus Flesh download this from the Cactus
%  homepage at www.cactuscode.org)
\usepackage{../../../../../doc/latex/cactus}

\newlength{\tableWidth} \newlength{\maxVarWidth} \newlength{\paraWidth} \newlength{\descWidth} \begin{document}

% The title of the document (not necessarily the name of the Thorn)
\title{Hydro\_RNSID}

% The author of the documentation
\author{Nik Stergioulas, Roberto De Pietri, Frank L\"ofller}

% the date your document was last changed:
\date{August 1 2017}

\maketitle

% Do not delete next line
% START CACTUS THORNGUIDE

% Add all definitions used in this documentation here
%   \def\mydef etc

% Add an abstract for this thorn's documentation
\begin{abstract}
Hydro\_RNSID - rotating relativistic neutron stars.
\end{abstract}

\section{Introduction}
\label{sec:intro}

This thorn generates neutron star initial data for the GRHydro code. As
with the Einstein Toolkit code itself, please feel free to add, alter 
or extend any part of this code. However please keep the documentation up to
date (even, or especially, if it's just to say what doesn't work).

This thorn effectively takes the public domain code {\tt RNSID}
written by Nik Stergioulas and interpolates the output onto a
Cartesian grid. This porting is based on an initila porting to 
Whisky by Luca Baiotti and Ian Hawke and has been adapted to 
GRHydro and Einstein Toolkit.


\section{RNSID}
\label{sec:rnsid}

RNSID, or rotating neutron star initial data, is a code based on the
Komatsu-Eriguch-Hachisu (KEH) method for constructing models of
rotating neutron stars. It allows for polytropic or tabulated
equations of state. For more details of the how the code works
see~\cite{rnsid1}, \cite{nsthesis} (appendix A is particularly
helpful) or especially \cite{nslivrev} which is the most up to date
and lists other possible methods of constructing rotating neutron star
initial data.

In short Hydro\_RNSID is a thorn that generate initial model for rotating 
isolated stars described by a zero-temperature tabulated Equation of State or
an iso-entrophic politropic EOS. The activation of the thorn for
genereting ID (The thorns ``Hydro\_Base'' and  ``GRHydro'' are the two prerequisites)

The model are generated specifing the central baryonic density
(rho\_central), the oblatness of the Star (axes\_ratio) and the
rotational profile (rotation\_type).  Currently two kinds of
rotational profiles are implemented: ``uniform'' for uniformly
rotating stars and ``diff'' for differentially rotating stars,
described by the j-law profile (parametrized by the parameter
A\_diff=$\hat{A}$):
\begin{equation}
  \Omega_c-\Omega = \frac{1}{\hat{A}^2 r_e^2}
  \left[ \frac{(\Omega-\omega) r^2 \sin^2 \theta e^{-2\nu}}{1-(\Omega-\omega)^2 r^2 \sin^2 \theta e^{-2\nu}}\right]
\end{equation}
where $r_e$ is the equatorial radius of the star and $\Omega$ is the
rotational angular velocity $\Omega=u^\phi/u^0$ and $\Omega_c$ is
$\Omega$ at the center of the star.


\section{Parameters of Thorn}
\label{sec:rnsid_par}

Here one can find definition of the main parameter the determine the
behaviour of the Thorn. The activation of the RNSID initial data
is achieved by the following line:
\begin{verbatim}
ActiveThorns="Hydro_Base GRHydro Hydro_RNSID"
#####
##### Setting for activating the ID 
#####
ADMBase::initial_data  = "hydro_rnsid"
ADMBase::initial_lapse = "hydro_rnsid"
ADMBase::initial_shift = "hydro_rnsid"
\end{verbatim}

The correspongig section of the parameter file is:
\begin{verbatim}
#####
##### Basic Setting
#####
Hydro_rnsid::rho_central   = 1.28e-3  # central baryon density (G=c=1) 
Hydro_rnsid::axes_ratio    = 1        # radial/equatorial axes ratio 
Hydro_rnsid::rotation_type = diff     # uniform = uniform rotation
Hydro_rnsid::A_diff        = 1        # Parameter of the diff rot-law.
Hydro_rnsid::accuracy      = 1e-10    # accuracy goal for convergence
\end{verbatim}



Than a section for setting the Equation of State (EOS) should be added.
If this section is missing a ``poly'' EOS will be used with default parameters.
The two possibilities are:
\begin{description}
\item{\textbf{Isentropic Polytrope:}}
  
  In this case the base setting for the initial data are specified
  giving the following parameters:
\begin{verbatim}
#####
##### Setting for polytrope
#####
Hydro_rnsid::eos_type  = "poly"  
Hydro_rnsid::RNS_Gamma = 2.0
Hydro_rnsid::RNS_K     = 165   
\end{verbatim}
They correspond at the following implementation of the EOS
that it is consistent with the 1st Law of thermodinamics.
\begin{eqnarray}
   p         &=& K \cdot \rho^\Gamma \\
   \epsilon &=& \frac{K}{\Gamma-1} \cdot \rho^{\Gamma-1} 
\end{eqnarray}
and for the above choice of parameters corresponding to the choice $K=165$ (in units where $G=c=M_\odot=1$) and $\Gamma=2$.

\item{\textbf{Tabulated EOS:}}

  In this case the (cold) EOS used to
  generate the initial data is read from a file

\begin{verbatim}
#####
##### Setting for tabulated EOS
#####
Hydro_rnsid::eos_type  = "tab"   
Hydro_rnsid::eos_file  = "full_path_name_of_the tabulated_EOS_file"
\end{verbatim}
  The syntax of the tabulated file is the same as for the original
  RNSID program and assumes that all quantities are expressed in the
  cgs system of units. The first line contains the number of tabulated
  values ($N$) while the next $N$ lines contain the values:
  $e=\rho(1+\varepsilon)$, $p$, $\log h = c^2 \log_e((e+p)/\rho)$, and
  $\rho$, respectively.
\end{description}


An additional section allows one to start initial data from a
previously generated binary file or to save the data generated at this
time. Usually the best way to proceed is to specify where the initial
data file should be located.
\begin{verbatim}
#####
##### Setting for recover and saving of 2d models 
#####
Hydro_rnsid::save_2Dmodel    = "yes"   # other possibility is no (default)
Hydro_rnsid::recover_2Dmodel = "yes"   # other possibility is no (default)
Hydro_rnsid::model2D_file    = "full_file_name"
\end{verbatim}

For examples of initial data generated using RNSID and their
evolutions, see~\cite{rotns1,rotns2}. In the par directory, examples
are provided as a perl file that produces the corresponding Cactus par
files. These examples correspond to the evolutions described
in~\cite{rotns1,rotns2}.

\section{Utility program}

Together with the Thorn, we distribute a self-executable version of
the initial data routine RNSID that accepts the same parameters as the
thorn and is able to create a binary file of the 2d initial data that
can be directly imported into the evolution code. Moreover, the
program RNS\_readID is provided that reads a 2d initial data file and
produces an hdf5 version of the data interpolated onto a 3d grid.


\begin{thebibliography}{1}

\bibitem{rotns1}
J.~A. Font, N. Stergioulas and K.~D. Kokkotas.
\newblock Nonlinear hydrodynamical evolution of rotating relativistic
stars: Numerical methods and code tests.
\newblock {\em Mon. Not. Roy. Astron. Soc.}, {\bf 313}, 678, 2000.

\bibitem{rotns2}
F.~L\"offler, R.~De Pietri, A.~Feo, F.~Maione and L.~Franci, 
\newblock Stiffness effects on the dynamics of the bar-mode instability 
of neutron stars in full general relativity. 
\newblock {\em Phys. Rev.}, {\bf D 91}, 064057, 2015 (arXiv:1411.1963).

\bibitem{rnsid1}
N. Stergioulas and J.~L. Friedmann.
\newblock Comparing models of rapidly rotating relativistic stars
constructed by two numerical methods.
\newblock {\em ApJ.}, {\bf 444}, 306, 1995.

\bibitem{nsthesis}
N. Stergioulas.
\newblock The structure and stability of rotating relativistic stars.
\newblock PhD thesis, University of Wisconsin-Milwaukee, 1996.

\bibitem{nslivrev}
N. Stergioulas.
\newblock Rotating Stars in Relativity
\newblock {\em Living Rev. Relativity}, {\bf 1}, 1998.
\newblock [Article in online journal], cited on 18/3/02,
  http://www.livingreviews.org/Articles/Volume1/1998-8stergio/index.html.

\end{thebibliography}

% Do not delete next line
% END CACTUS THORNGUIDE



\section{Parameters} 


\parskip = 0pt

\setlength{\tableWidth}{160mm}

\setlength{\paraWidth}{\tableWidth}
\setlength{\descWidth}{\tableWidth}
\settowidth{\maxVarWidth}{rns\_atmo\_tolerance}

\addtolength{\paraWidth}{-\maxVarWidth}
\addtolength{\paraWidth}{-\columnsep}
\addtolength{\paraWidth}{-\columnsep}
\addtolength{\paraWidth}{-\columnsep}

\addtolength{\descWidth}{-\columnsep}
\addtolength{\descWidth}{-\columnsep}
\addtolength{\descWidth}{-\columnsep}
\noindent \begin{tabular*}{\tableWidth}{|c|l@{\extracolsep{\fill}}r|}
\hline
\multicolumn{1}{|p{\maxVarWidth}}{a\_diff} & {\bf Scope:} private & REAL \\\hline
\multicolumn{3}{|p{\descWidth}|}{{\bf Description:}   {\em constant A in differential rotation law}} \\
\hline{\bf Range} & &  {\bf Default:} 1.0 \\\multicolumn{1}{|p{\maxVarWidth}|}{\centering 0.0:} & \multicolumn{2}{p{\paraWidth}|}{Any positive number} \\\hline
\end{tabular*}

\vspace{0.5cm}\noindent \begin{tabular*}{\tableWidth}{|c|l@{\extracolsep{\fill}}r|}
\hline
\multicolumn{1}{|p{\maxVarWidth}}{accuracy} & {\bf Scope:} private & REAL \\\hline
\multicolumn{3}{|p{\descWidth}|}{{\bf Description:}   {\em rnsid accuracy in convergence }} \\
\hline{\bf Range} & &  {\bf Default:} 1.0e-7 \\\multicolumn{1}{|p{\maxVarWidth}|}{\centering 0:} & \multicolumn{2}{p{\paraWidth}|}{Any positive number} \\\hline
\end{tabular*}

\vspace{0.5cm}\noindent \begin{tabular*}{\tableWidth}{|c|l@{\extracolsep{\fill}}r|}
\hline
\multicolumn{1}{|p{\maxVarWidth}}{axes\_ratio} & {\bf Scope:} private & REAL \\\hline
\multicolumn{3}{|p{\descWidth}|}{{\bf Description:}   {\em rnsid axes ratio}} \\
\hline{\bf Range} & &  {\bf Default:} 1 \\\multicolumn{1}{|p{\maxVarWidth}|}{\centering 0:} & \multicolumn{2}{p{\paraWidth}|}{Any positive number} \\\hline
\end{tabular*}

\vspace{0.5cm}\noindent \begin{tabular*}{\tableWidth}{|c|l@{\extracolsep{\fill}}r|}
\hline
\multicolumn{1}{|p{\maxVarWidth}}{cf} & {\bf Scope:} private & REAL \\\hline
\multicolumn{3}{|p{\descWidth}|}{{\bf Description:}   {\em Convergence factor}} \\
\hline{\bf Range} & &  {\bf Default:} 1.0 \\\multicolumn{1}{|p{\maxVarWidth}|}{\centering 0:} & \multicolumn{2}{p{\paraWidth}|}{Any positive number} \\\hline
\end{tabular*}

\vspace{0.5cm}\noindent \begin{tabular*}{\tableWidth}{|c|l@{\extracolsep{\fill}}r|}
\hline
\multicolumn{1}{|p{\maxVarWidth}}{eos\_file} & {\bf Scope:} private & STRING \\\hline
\multicolumn{3}{|p{\descWidth}|}{{\bf Description:}   {\em Equation of state table}} \\
\hline{\bf Range} & &  {\bf Default:} (none) \\\multicolumn{1}{|p{\maxVarWidth}|}{\centering .*} & \multicolumn{2}{p{\paraWidth}|}{EOS table file} \\\hline
\end{tabular*}

\vspace{0.5cm}\noindent \begin{tabular*}{\tableWidth}{|c|l@{\extracolsep{\fill}}r|}
\hline
\multicolumn{1}{|p{\maxVarWidth}}{eos\_type} & {\bf Scope:} private & KEYWORD \\\hline
\multicolumn{3}{|p{\descWidth}|}{{\bf Description:}   {\em Specify type of equation of state}} \\
\hline{\bf Range} & &  {\bf Default:} poly \\\multicolumn{1}{|p{\maxVarWidth}|}{\centering poly} & \multicolumn{2}{p{\paraWidth}|}{Polytropic EOS} \\\multicolumn{1}{|p{\maxVarWidth}|}{\centering tab} & \multicolumn{2}{p{\paraWidth}|}{Tabulated EOS} \\\hline
\end{tabular*}

\vspace{0.5cm}\noindent \begin{tabular*}{\tableWidth}{|c|l@{\extracolsep{\fill}}r|}
\hline
\multicolumn{1}{|p{\maxVarWidth}}{model2d\_file} & {\bf Scope:} private & STRING \\\hline
\multicolumn{3}{|p{\descWidth}|}{{\bf Description:}   {\em Name of 2D model file}} \\
\hline{\bf Range} & &  {\bf Default:} model2D.dat \\\multicolumn{1}{|p{\maxVarWidth}|}{\centering .*} & \multicolumn{2}{p{\paraWidth}|}{Default 2D model file} \\\hline
\end{tabular*}

\vspace{0.5cm}\noindent \begin{tabular*}{\tableWidth}{|c|l@{\extracolsep{\fill}}r|}
\hline
\multicolumn{1}{|p{\maxVarWidth}}{recover\_2dmodel} & {\bf Scope:} private & KEYWORD \\\hline
\multicolumn{3}{|p{\descWidth}|}{{\bf Description:}   {\em Recover 2D model?}} \\
\hline{\bf Range} & &  {\bf Default:} no \\\multicolumn{1}{|p{\maxVarWidth}|}{\centering yes} & \multicolumn{2}{p{\paraWidth}|}{recover 2D model} \\\multicolumn{1}{|p{\maxVarWidth}|}{\centering no} & \multicolumn{2}{p{\paraWidth}|}{don't recover 2D model} \\\hline
\end{tabular*}

\vspace{0.5cm}\noindent \begin{tabular*}{\tableWidth}{|c|l@{\extracolsep{\fill}}r|}
\hline
\multicolumn{1}{|p{\maxVarWidth}}{rho\_central} & {\bf Scope:} private & REAL \\\hline
\multicolumn{3}{|p{\descWidth}|}{{\bf Description:}   {\em Central Density for Star}} \\
\hline{\bf Range} & &  {\bf Default:} 1.24e-3 \\\multicolumn{1}{|p{\maxVarWidth}|}{\centering :} & \multicolumn{2}{p{\paraWidth}|}{} \\\hline
\end{tabular*}

\vspace{0.5cm}\noindent \begin{tabular*}{\tableWidth}{|c|l@{\extracolsep{\fill}}r|}
\hline
\multicolumn{1}{|p{\maxVarWidth}}{rns\_atmo\_tolerance} & {\bf Scope:} private & REAL \\\hline
\multicolumn{3}{|p{\descWidth}|}{{\bf Description:}   {\em A point is set to atmosphere if rho {\textless} (1+RNS\_atmo\_tolerance)*RNS\_rho\_min}} \\
\hline{\bf Range} & &  {\bf Default:} 0.00001 \\\multicolumn{1}{|p{\maxVarWidth}|}{\centering 0.0:} & \multicolumn{2}{p{\paraWidth}|}{Zero or larger. A useful value could be 0.0001} \\\hline
\end{tabular*}

\vspace{0.5cm}\noindent \begin{tabular*}{\tableWidth}{|c|l@{\extracolsep{\fill}}r|}
\hline
\multicolumn{1}{|p{\maxVarWidth}}{rns\_gamma} & {\bf Scope:} private & REAL \\\hline
\multicolumn{3}{|p{\descWidth}|}{{\bf Description:}   {\em If we're using a different EoS at run time, this is the RNS Gamma}} \\
\hline{\bf Range} & &  {\bf Default:} 2 \\\multicolumn{1}{|p{\maxVarWidth}|}{\centering *:*} & \multicolumn{2}{p{\paraWidth}|}{Will be ignored if negative} \\\hline
\end{tabular*}

\vspace{0.5cm}\noindent \begin{tabular*}{\tableWidth}{|c|l@{\extracolsep{\fill}}r|}
\hline
\multicolumn{1}{|p{\maxVarWidth}}{rns\_k} & {\bf Scope:} private & REAL \\\hline
\multicolumn{3}{|p{\descWidth}|}{{\bf Description:}   {\em If we're using a different EoS at run time, this is the RNS K}} \\
\hline{\bf Range} & &  {\bf Default:} 100 \\\multicolumn{1}{|p{\maxVarWidth}|}{\centering *:*} & \multicolumn{2}{p{\paraWidth}|}{Will be ignored if negative} \\\hline
\end{tabular*}

\vspace{0.5cm}\noindent \begin{tabular*}{\tableWidth}{|c|l@{\extracolsep{\fill}}r|}
\hline
\multicolumn{1}{|p{\maxVarWidth}}{rns\_lmax} & {\bf Scope:} private & INT \\\hline
\multicolumn{3}{|p{\descWidth}|}{{\bf Description:}   {\em max. term in Legendre poly.}} \\
\hline{\bf Range} & &  {\bf Default:} 10 \\\multicolumn{1}{|p{\maxVarWidth}|}{\centering 1:} & \multicolumn{2}{p{\paraWidth}|}{Any positive, non zero number} \\\hline
\end{tabular*}

\vspace{0.5cm}\noindent \begin{tabular*}{\tableWidth}{|c|l@{\extracolsep{\fill}}r|}
\hline
\multicolumn{1}{|p{\maxVarWidth}}{rns\_rho\_min} & {\bf Scope:} private & REAL \\\hline
\multicolumn{3}{|p{\descWidth}|}{{\bf Description:}   {\em A minimum rho below which evolution is turned off (atmosphere).}} \\
\hline{\bf Range} & &  {\bf Default:} 1.0e-14 \\\multicolumn{1}{|p{\maxVarWidth}|}{\centering 0.0:} & \multicolumn{2}{p{\paraWidth}|}{Atmosphere detection for RNSID} \\\hline
\end{tabular*}

\vspace{0.5cm}\noindent \begin{tabular*}{\tableWidth}{|c|l@{\extracolsep{\fill}}r|}
\hline
\multicolumn{1}{|p{\maxVarWidth}}{rotation\_type} & {\bf Scope:} private & KEYWORD \\\hline
\multicolumn{3}{|p{\descWidth}|}{{\bf Description:}   {\em Specify type of rotation law}} \\
\hline{\bf Range} & &  {\bf Default:} uniform \\\multicolumn{1}{|p{\maxVarWidth}|}{\centering uniform} & \multicolumn{2}{p{\paraWidth}|}{uniform rotation} \\\multicolumn{1}{|p{\maxVarWidth}|}{\centering diff} & \multicolumn{2}{p{\paraWidth}|}{KEH differential rotation law} \\\hline
\end{tabular*}

\vspace{0.5cm}\noindent \begin{tabular*}{\tableWidth}{|c|l@{\extracolsep{\fill}}r|}
\hline
\multicolumn{1}{|p{\maxVarWidth}}{save\_2dmodel} & {\bf Scope:} private & KEYWORD \\\hline
\multicolumn{3}{|p{\descWidth}|}{{\bf Description:}   {\em Save 2D model?}} \\
\hline{\bf Range} & &  {\bf Default:} no \\\multicolumn{1}{|p{\maxVarWidth}|}{\centering yes} & \multicolumn{2}{p{\paraWidth}|}{save 2D model} \\\multicolumn{1}{|p{\maxVarWidth}|}{\centering no} & \multicolumn{2}{p{\paraWidth}|}{don't save 2D model} \\\hline
\end{tabular*}

\vspace{0.5cm}\noindent \begin{tabular*}{\tableWidth}{|c|l@{\extracolsep{\fill}}r|}
\hline
\multicolumn{1}{|p{\maxVarWidth}}{zero\_shift} & {\bf Scope:} private & KEYWORD \\\hline
\multicolumn{3}{|p{\descWidth}|}{{\bf Description:}   {\em Set shift to zero?}} \\
\hline{\bf Range} & &  {\bf Default:} no \\\multicolumn{1}{|p{\maxVarWidth}|}{\centering yes} & \multicolumn{2}{p{\paraWidth}|}{set shift to zero} \\\multicolumn{1}{|p{\maxVarWidth}|}{\centering no} & \multicolumn{2}{p{\paraWidth}|}{don't set shift to zero} \\\hline
\end{tabular*}

\vspace{0.5cm}\noindent \begin{tabular*}{\tableWidth}{|c|l@{\extracolsep{\fill}}r|}
\hline
\multicolumn{1}{|p{\maxVarWidth}}{initial\_hydro} & {\bf Scope:} shared from HYDROBASE & KEYWORD \\\hline
\multicolumn{3}{|l|}{\bf Extends ranges:}\\ 
\hline\multicolumn{1}{|p{\maxVarWidth}|}{\centering hydro\_rnsid} & \multicolumn{2}{p{\paraWidth}|}{Construnct stationary initial data with rnsid} \\\hline
\end{tabular*}

\vspace{0.5cm}\noindent \begin{tabular*}{\tableWidth}{|c|l@{\extracolsep{\fill}}r|}
\hline
\multicolumn{1}{|p{\maxVarWidth}}{timelevels} & {\bf Scope:} shared from HYDROBASE & INT \\\hline
\end{tabular*}

\vspace{0.5cm}\parskip = 10pt 

\section{Interfaces} 


\parskip = 0pt

\vspace{3mm} \subsection*{General}

\noindent {\bf Implements}: 

hydro\_rnsid
\vspace{2mm}

\noindent {\bf Inherits}: 

admbase

hydrobase
\vspace{2mm}

\vspace{5mm}

\noindent {\bf Uses header}: 

FishEye.h

Boundary.h

Symmetry.h
\vspace{2mm}\parskip = 10pt 

\section{Schedule} 


\parskip = 0pt


\noindent This section lists all the variables which are assigned storage by thorn EinsteinInitialData/Hydro\_RNSID.  Storage can either last for the duration of the run ({\bf Always} means that if this thorn is activated storage will be assigned, {\bf Conditional} means that if this thorn is activated storage will be assigned for the duration of the run if some condition is met), or can be turned on for the duration of a schedule function.


\subsection*{Storage}

\hspace{5mm}

 \begin{tabular*}{160mm}{ll} 
~& {\bf Conditional:} \\ 
~ &  ADMBase::metric[2] ADMBase::curv[2] ADMBase::lapse[2] ADMBase::shift[2]\\ 
~ & ~\\ 
\end{tabular*} 


\subsection*{Scheduled Functions}
\vspace{5mm}

\noindent {\bf CCTK\_PARAMCHECK}   (conditional) 

\hspace{5mm} hydro\_rnsid\_checkparameters 

\hspace{5mm}{\it check parameters } 


\hspace{5mm}

 \begin{tabular*}{160mm}{cll} 
~ & Language:  & c \\ 
~ & Type:  & function \\ 
\end{tabular*} 


\vspace{5mm}

\noindent {\bf HydroBase\_Initial}   (conditional) 

\hspace{5mm} hydro\_rnsid\_init 

\hspace{5mm}{\it create rotating neutron star initial data } 


\hspace{5mm}

 \begin{tabular*}{160mm}{cll} 
~ & Language:  & c \\ 
~ & Sync:  & admbase::metric \\ 
~& ~ &admbase::curv\\ 
~& ~ &admbase::lapse\\ 
~& ~ &admbase::shift\\ 
~& ~ &hydrobase::rho\\ 
~& ~ &hydrobase::press\\ 
~& ~ &hydrobase::eps\\ 
~& ~ &hydrobase::vel\\ 
~ & Type:  & function \\ 
\end{tabular*} 



\vspace{5mm}\parskip = 10pt 
\end{document}
