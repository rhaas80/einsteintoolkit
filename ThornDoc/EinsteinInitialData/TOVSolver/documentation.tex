% *======================================================================*
%  Cactus Thorn template for ThornGuide documentation
%  Author: Ian Kelley
%  Date: Sun Jun 02, 2002
%
%  Thorn documentation in the latex file doc/documentation.tex
%  will be included in ThornGuides built with the Cactus make system.
%  The scripts employed by the make system automatically include
%  pages about variables, parameters and scheduling parsed from the
%  relevant thorn CCL files.
%
%  This template contains guidelines which help to assure that your
%  documentation will be correctly added to ThornGuides. More
%  information is available in the Cactus UsersGuide.
%
%  Guidelines:
%   - Do not change anything before the line
%       % START CACTUS THORNGUIDE",
%     except for filling in the title, author, date, etc. fields.
%        - Each of these fields should only be on ONE line.
%        - Author names should be separated with a \\ or a comma.
%   - You can define your own macros, but they must appear after
%     the START CACTUS THORNGUIDE line, and must not redefine standard
%     latex commands.
%   - To avoid name clashes with other thorns, 'labels', 'citations',
%     'references', and 'image' names should conform to the following
%     convention:
%       ARRANGEMENT_THORN_LABEL
%     For example, an image wave.eps in the arrangement CactusWave and
%     thorn WaveToyC should be renamed to CactusWave_WaveToyC_wave.eps
%   - Graphics should only be included using the graphicx package.
%     More specifically, with the "\includegraphics" command.  Do
%     not specify any graphic file extensions in your .tex file. This
%     will allow us to create a PDF version of the ThornGuide
%     via pdflatex.
%   - References should be included with the latex "\bibitem" command.
%   - Use \begin{abstract}...\end{abstract} instead of \abstract{...}
%   - Do not use \appendix, instead include any appendices you need as
%     standard sections.
%   - For the benefit of our Perl scripts, and for future extensions,
%     please use simple latex.
%
% *======================================================================*
%
% Example of including a graphic image:
%    \begin{figure}[ht]
% 	\begin{center}
%    	   \includegraphics[width=6cm]{/home/runner/work/einsteintoolkit/einsteintoolkit/arrangements/EinsteinInitialData/TOVSolver/doc/MyArrangement_MyThorn_MyFigure}
% 	\end{center}
% 	\caption{Illustration of this and that}
% 	\label{MyArrangement_MyThorn_MyLabel}
%    \end{figure}
%
% Example of using a label:
%   \label{MyArrangement_MyThorn_MyLabel}
%
% Example of a citation:
%    \cite{MyArrangement_MyThorn_Author99}
%
% Example of including a reference
%   \bibitem{MyArrangement_MyThorn_Author99}
%   {J. Author, {\em The Title of the Book, Journal, or periodical}, 1 (1999),
%   1--16. {\tt http://www.nowhere.com/}}
%
% *======================================================================*

% If you are using CVS use this line to give version information


\documentclass{article}


% Use the Cactus ThornGuide style file
% (Automatically used from Cactus distribution, if you have a
%  thorn without the Cactus Flesh download this from the Cactus
%  homepage at www.cactuscode.org)
\usepackage{../../../../../doc/latex/cactus}


\newlength{\tableWidth} \newlength{\maxVarWidth} \newlength{\paraWidth} \newlength{\descWidth} \begin{document}

% The author of the documentation
\author{I.~Hawke,\\
 F.~Loeffler \textless loeffler@sissa.it \textgreater } 
%A.~Nagar \textless alessandro.nagar@polito.it\textgreater}

% The title of the document (not necessarily the name of the Thorn)
\title{TOVSolver}

% the date your document was last changed, if your document is in CVS,
% please use:
%    \date{$ $Date: 2009-09-17 15:39:33 -0500 (Thu, 17 Sep 2009) $ $}

\date{\today}

\maketitle

% Do not delete next line
% START CACTUS THORNGUIDE

% Add all definitions used in this documentation here
%   \def\mydef etc

\begin{abstract}
  This thorn solves the Tolman-Oppenheimer-Volkov equations of hydrostatic equilibrium
  for a spherically symmetric static star. 
\end{abstract}


%---------------------
\section{Introduction}
\label{sec:intro}
%---------------------

The Tolman-Oppenheimer-Volkoff solution is a static perfect fluid
``star''. It is frequently used as a test of relativistic hydro
codes. Here it is intended for use without evolving the matter
terms. This provides a compact strong field solution which is static
but does not contain singularities.


%------------------
\section{Equations}
\label{sec:eqn}
%------------------

The equations for a TOV star~\cite{Tolman39,OppVol39,mtw} are usually
derived in Schwarzschild coordinates. In these coordinates, the metric can
be brought into the form
\begin{equation}
ds^2 = -e^{2\phi}dt^2 + \left(1-\dfrac{2m}{r}\right)^{-1}dr^2 + r^2 d\Omega^2 \ .
\end{equation}


This thorn is based on the notes of Thomas Baumgarte~\cite{Baumgarte-file} that 
have been partially included in this documentation. However the notation for the
fluid quantities follows~\cite{Font00a}. 
Here we are assuming that the stress energy tensor is given by
\begin{equation}
  \label{eq:Tmunu}
  T^{\mu\nu} = (\mu + P)u^{\mu}u^{\nu} + Pg^{\mu\nu},
\end{equation}
where $\mu$ is the total energy, $P$ the pressure, $u^{\mu}$ the fluid
four velocity, $\rho$ the rest-mass density, $\epsilon$ the specific
internal energy, and
\begin{eqnarray}
  \label{eq:fluidquantities}
  \mu & = & \rho (1 + \epsilon), \\
  P & = & (\Gamma - 1)\rho\epsilon, \\
  P & = & K \rho^{\Gamma}.
\end{eqnarray}
This enforces a polytropic equation of state. We note that in Cactus
the units are $c = G = M_{\odot} = 1$.

The equations to give the initial data are solved (as usual) in the
Schwarzschild-like coordinates with the areal radius labelled $r$.
The equations of the relativistic hydrostatic equilibrium are
\begin{eqnarray}
  \label{eq:TOViso}
  \frac{d P}{d r} & = & -(\mu + P) \frac{m + 4\pi r^3 P}{r(r - 2m)}, \\
%
  \frac{d m}{d r} & = & 4 \pi r^2 \mu, \\
%
  \frac{d \phi}{d r} & = & \frac{m + 4\pi r^3 P}{r(r -
    2m)} \ . \\
\end{eqnarray}
Here $m$ is the gravitational mass inside the sphere radius $r$, and
$\phi$ the logarithm of the lapse. Once the integration is done for the 
interior of the star we match to the exterior (see below). In the exterior 
we have
\begin{align}
  \label{eq:TOVexterior}
     P & =  {\tt TOV\_atmosphere}, \\
     m & =  M, \\
  \phi & = \dfrac{1}{2} \log(1-2M / r).
\end{align}

In order to impose initial data in cartesian coordinates, we want to transform
this solution to isotropic coordinates, in which the metric takes the form
\begin{equation}
\label{eq:metr_iso}
ds^2 = -e^{2\phi}dt^2+e^{2\Lambda}\left(d\bar{r}^2+\bar{r}^2d\Omega^2\right) \ .
\end{equation}
Here $\bar{r}$ denotes the isotropic radius. Matching the two metrics, one
obviously finds
\begin{align}
r^2                                    &= e^{2\Lambda}\bar{r}^2 \ , \\
\left(1-\dfrac{2m}{r}\right)^{-1} dr^2 &= e^{2\Lambda}d\bar{r}^2 \ .
\end{align}
As a result, we have an additional differential equation to solve in order
to have $\bar{r}(r)$, that is
\begin{equation}
\label{eq:rbar}
\frac{d (\log(\bar{r} / r))}{\partial r} =  \frac{r^{1/2} - (r-2m)^{1/2}}{r(r-2m)^{1/2}} \ .
\end{equation}
Given such a solution, the missing metric potential is simply given by
\begin{equation}
e^{\Lambda} = \dfrac{r}{\bar{r}} \ .
\end{equation}
In the following section we concentrate on solving Eq.~(\ref{eq:rbar}) in the
exterior and in the interior of the star.

Then, given these one-dimensional data we interpolate to get data on 
the three-dimensional Cactus grid; that is, we interpolate on the three dimensional 
{\tt r} given by the {\tt x, y, z} variables the physical hydro and spacetime
quantities that are function of the isotropic radius $\bar{r}$ computed above. 
Only linear interpolation is used. This avoids problems at the surface of the 
star, and does not cause problems if the number of points in the one dimensional 
array is sufficient ($1\times 10^5$ is the default, which should be sufficient for 
medium-sized grids).

%--------------------
\subsection{Exterior}
\label{sbsc:exterior}
%--------------------

In the exterior of the star, $r>R$, the mass $M\equiv m(R)$ is constant, and 
Eq.~(\ref{eq:rbar}) can be solved analytically up to a constant of integration. 
Fixing this constant such that $r$ and $\bar{r}$ agree at infinity, we find
\begin{equation}
\bar{r} = \dfrac{1}{2}\left(\sqrt{r^2-2Mr}+r -M\right) \ ,
\end{equation}
or, solving for $r$ [cfr. Exercise 31.7 of MTW~\cite{mtw}]
\begin{equation}
r=\bar{r}\left(1+\dfrac{M}{2\bar{r}}\right)^2 \ .
\end{equation}
The metric potential as a function of $\bar{r}$ is obviously
\begin{equation}
e^{2\Lambda} = \left(1+\dfrac{M}{2\bar{r}}\right)^2 \ .
\end{equation}

%--------------------
\subsection{Interior}
\label{sbsc:interior}
%--------------------

In the interior, Eq,~(\ref{eq:rbar}) can not be integrated analytically, because
$m$ is now a function of $r$. Instead, we have to integrate
\begin{equation}
\int_0^{\bar{r}} \dfrac{d\bar{r}}{\bar{r}} = \int_0^r\left(1-\dfrac{2m}{r}\right)^{-1/2}\dfrac{dr}{r} \ .
\end{equation}
The left hand side can be integrated analytically, and has a singular point at
$\bar{r}=0$. The right hand side cannot be integrated analytically, but will also
be singular at $r=0$, which poses problems when trying to integrate the equation
numerically. We therefore rewrite the right hand side by adding and subracting
a term $1/r$, which yields
\begin{equation}
\int_0^r\dfrac{1}{r(1-2m/r)^{1/2}}dr = \int_0^r\dfrac{1-(1-2m/r)^{1/2}}{r(1-2m/r)^{1/2}}dr+\int_0^r\dfrac{dr}{r} \ .
\end{equation}
Since $m\sim r^3$ close to the origin, the first term on the right hand side is now
regular and the second one can be integrated analytically. As a result, we find
\begin{equation}
\int_0^{\bar{r}}d\ln\bar{r}-\int_0^rd\ln r=\int_0^r\dfrac{1-(1-2m/r)^{1/2}}{r(1-2m/r)^{1/2}}dr \ .
\end{equation}
Replacing the lower limits ($r=\bar{r}=0$) temporarily with $r_0$ and $\bar{r}_0$, we can integrate
the right hand side and find
\begin{equation}
\ln\left(\dfrac{\bar{r}}{r}\right)-\ln\left(\dfrac{\bar{r}_0}{r_0}\right)=\int_0^r\dfrac{1-(1-2m/r)^{1/2}}{r(1-2m/r)^{1/2}}dr \ ,
\end{equation}
or
\begin{equation}
\bar{r} = C r \exp\left[\int_0^r\dfrac{1-(1-2m/r)^{1/2}}{r(1-2m/r)^{1/2}}dr\right] \ .
\end{equation}
Here the constant of integration $C$ is related to the ratio $\bar{r}_0/r_0$ evaluated at
the origin (which is perfectly regular). It can be chosen such that the interior solution
matches the exterior solution at the surface of the star. This requirement implies
\begin{equation}
C = \dfrac{1}{2R}\left(\sqrt{R^2-2MR}+R-M\right)\exp\left[-\int_0^R\dfrac{1-(1-2m/r)^{1/2}}{r(1-2m/r)^{1/2}}dr\right] \ .
\end{equation}
In this respect, let us recall how we do initial data for the system of 
equations at $r=0$.  Given a value of the central density $\rho_c$ (in
Cactus units) we pose $P_0 = K\rho_c^{\Gamma}$, $m_0=0$, $\phi_0=0$ and 
$\bar{r}_0 = {\tt TOV\_Tiny}$, $r_0 = {\tt TOV\_Tiny}$. The {\tt TOV\_Tiny} 
number is hardwired into the code to avoid divide by zero errors; it is $10^{-20}$. 
Also the default parameters will give the TOV star used for the long term evolutions
in~\cite{Font02a}. That is, a nonrotating $N=1$ ($\gamma=1+1/N=2$) polytropic star 
with gravitational mass $M=1.4M_{\odot}$, circumferential radius $R=14.15$km, central 
rest-mass density $\rho_c=1.28 \times 10^{-3}$ and $K=100$.


%--------------------------
\section{Use of this thorn}
\label{sec:use}
%--------------------------

To use this thorn to provide initial data for the {\tt ADMBase}
variables $\alpha$, $\beta$, $g$ and $K$ just activate the thorn and
set {\tt ADMBase:initial\_data = ``TOV''}.

There are two ways of coupling the matter sources to the thorn that
evolves the Einstein equations. One is to use the {\tt CalcTmunu}
interface. This will give the components of the stress energy tensor
\emph{pointwise} across the grid. For an example of this, see thorn
{\tt ADM} in {\tt CactusEinstein}.

To use the {\tt CalcTmunu} interface you should
\begin{itemize}
\item put the lines
\begin{verbatim}
friend: ADMCoupling

USES INCLUDE: CalcTmunu.inc
USES INCLUDE: CalcTmunu_temps.inc
USES INCLUDE: CalcTmunu_rfr.inc
\end{verbatim}
in your {\tt interface.ccl}
\item In any routine requiring the matter terms, put 
  \begin{itemize}
  \item {\tt \#include ``CalcTmunu\_temps.inc''} in the variable
    declarations
  \item declare {\tt CCTK\_REAL}s {\tt Ttt, Ttx, Tty, Ttz, Txx, Txy,
      Txz, Tyy, Tyz, Tzz}.
  \item Inside an {\tt i,j,k} loop put {\tt \#include
      ``CalcTmunu.inc''}. {\bf This must be a Fortran routine} (We
    could probably fix this if requested).
  \end{itemize}
\item You then use the real numbers {\tt Ttt} etc.~as the stress
  energy tensor at a point.
\end{itemize}

As an alternative you can use the grid functions {\tt StressEnergytt,
  StressEnergytx}, etc.~directly to have the stress energy tensor over
the entire grid. To do this you just need the line {\tt friend:
  ADMCoupling} in your {\tt interface.ccl}. Although this seems much
simpler, you will now \emph{only} get the contributions from the {\tt
  TOVSolver} thorn. If you want to use other matter sources, most of
the current thorns ({\tt CosmologicalConstant}, the hydro code, the
scalar field code) all use the {\tt CalcTmunu} interface.

You also have the possibility to use a parameter
{\tt GRHydrotovsolver::TOV\_Separation} to obtain a spacetime consisting
of one TOV-system for $x>0$ and a second (similar) for $x<0$. This parameter
sets the separation of the centers of two neutron stars, has to be positive
and should be larger than twice the radius of one star.\\
Be aware that the spacetime obtained by this is no physical spacetime and
no solution of Einsteins Equations and therefore an IVP-run has to follow.
This parameter was only introduced for testing purposes of the IVP-Solver
and should only be considered as such. There would be better (and also easy)
ways to obtain initial data for two TOVs than that.





\begin{thebibliography}{10}

\bibitem{Tolman39}
R.~C. Tolman, Phys. Rev. {\bf 55}, 364 (1939).
%
\bibitem{OppVol39}
J.~R. Oppenheimer and G. Volkoff, Physical Review {\bf 55}, 374 (1939).
%
\bibitem{mtw}
C.W.~Misner, K.S.~Thorn and J.A.~Wheeler, Gravitation (Freeman and co. NY, 1973).
%
\bibitem{Baumgarte-file}
T.~W. Baumgarte. There is a copy of his notes in this directory: \\
TOVSolver/doc.
%
\bibitem{Font00a}
J.~A. Font, M. Miller, W. Suen and M. Tobias, Phys. Rev. {\bf D61},
044011 (2000).
%
\bibitem{Font02a}
J.~A. Font, T. Goodale, S. Iyer, M. Miller, L. Rezzolla, E. Seidel,
N. Stergioulas, W. Suen and M. Tobias, Phys. Rev. {\bf D65},
084024 (2002).


\end{thebibliography}

% Do not delete next line
% END CACTUS THORNGUIDE




\section{Parameters} 


\parskip = 0pt

\setlength{\tableWidth}{160mm}

\setlength{\paraWidth}{\tableWidth}
\setlength{\descWidth}{\tableWidth}
\settowidth{\maxVarWidth}{tov\_conformal\_flat\_three\_metric}

\addtolength{\paraWidth}{-\maxVarWidth}
\addtolength{\paraWidth}{-\columnsep}
\addtolength{\paraWidth}{-\columnsep}
\addtolength{\paraWidth}{-\columnsep}

\addtolength{\descWidth}{-\columnsep}
\addtolength{\descWidth}{-\columnsep}
\addtolength{\descWidth}{-\columnsep}
\noindent \begin{tabular*}{\tableWidth}{|c|l@{\extracolsep{\fill}}r|}
\hline
\multicolumn{1}{|p{\maxVarWidth}}{pert\_amplitude} & {\bf Scope:} private & REAL \\\hline
\multicolumn{3}{|p{\descWidth}|}{{\bf Description:}   {\em Amplitude of perturbation}} \\
\hline{\bf Range} & &  {\bf Default:} 0.0 \\\multicolumn{1}{|p{\maxVarWidth}|}{\centering *:*} & \multicolumn{2}{p{\paraWidth}|}{Anything} \\\hline
\end{tabular*}

\vspace{0.5cm}\noindent \begin{tabular*}{\tableWidth}{|c|l@{\extracolsep{\fill}}r|}
\hline
\multicolumn{1}{|p{\maxVarWidth}}{perturb} & {\bf Scope:} private & BOOLEAN \\\hline
\multicolumn{3}{|p{\descWidth}|}{{\bf Description:}   {\em Add density perturbation (you should solve the IVP if true)}} \\
\hline & & {\bf Default:} no \\\hline
\end{tabular*}

\vspace{0.5cm}\noindent \begin{tabular*}{\tableWidth}{|c|l@{\extracolsep{\fill}}r|}
\hline
\multicolumn{1}{|p{\maxVarWidth}}{tov\_clear\_initial\_data} & {\bf Scope:} private & BOOLEAN \\\hline
\multicolumn{3}{|p{\descWidth}|}{{\bf Description:}   {\em Clear initial data (spacetime)? Default is yes.}} \\
\hline & & {\bf Default:} yes \\\hline
\end{tabular*}

\vspace{0.5cm}\noindent \begin{tabular*}{\tableWidth}{|c|l@{\extracolsep{\fill}}r|}
\hline
\multicolumn{1}{|p{\maxVarWidth}}{tov\_combine\_method} & {\bf Scope:} private & KEYWORD \\\hline
\multicolumn{3}{|p{\descWidth}|}{{\bf Description:}   {\em Which combine method should be used.}} \\
\hline{\bf Range} & &  {\bf Default:} average \\\multicolumn{1}{|p{\maxVarWidth}|}{\centering maximum} & \multicolumn{2}{p{\paraWidth}|}{Take the maximum of rho and gxx as clue for the rest as clue.} \\\multicolumn{1}{|p{\maxVarWidth}|}{\centering average} & \multicolumn{2}{p{\paraWidth}|}{Take the average of all available parts.} \\\hline
\end{tabular*}

\vspace{0.5cm}\noindent \begin{tabular*}{\tableWidth}{|c|l@{\extracolsep{\fill}}r|}
\hline
\multicolumn{1}{|p{\maxVarWidth}}{tov\_conformal\_flat\_three\_metric} & {\bf Scope:} private & BOOLEAN \\\hline
\multicolumn{3}{|p{\descWidth}|}{{\bf Description:}   {\em Use conformal factor to get the 3-metric flat. default is no}} \\
\hline & & {\bf Default:} no \\\hline
\end{tabular*}

\vspace{0.5cm}\noindent \begin{tabular*}{\tableWidth}{|c|l@{\extracolsep{\fill}}r|}
\hline
\multicolumn{1}{|p{\maxVarWidth}}{tov\_dr} & {\bf Scope:} private & REAL \\\hline
\multicolumn{3}{|p{\descWidth}|}{{\bf Description:}   {\em The spacing in the radial direction on the 1d grid}} \\
\hline{\bf Range} & &  {\bf Default:} 5.e-4 \\\multicolumn{1}{|p{\maxVarWidth}|}{\centering (0.0:*} & \multicolumn{2}{p{\paraWidth}|}{Greater than 0} \\\hline
\end{tabular*}

\vspace{0.5cm}\noindent \begin{tabular*}{\tableWidth}{|c|l@{\extracolsep{\fill}}r|}
\hline
\multicolumn{1}{|p{\maxVarWidth}}{tov\_enforce\_interpolation} & {\bf Scope:} private & BOOLEAN \\\hline
\multicolumn{3}{|p{\descWidth}|}{{\bf Description:}   {\em Enforce the interpolation of the data onto the Hydro GFs even without tov as specified initial data}} \\
\hline & & {\bf Default:} no \\\hline
\end{tabular*}

\vspace{0.5cm}\noindent \begin{tabular*}{\tableWidth}{|c|l@{\extracolsep{\fill}}r|}
\hline
\multicolumn{1}{|p{\maxVarWidth}}{tov\_fake\_evolution} & {\bf Scope:} private & INT \\\hline
\multicolumn{3}{|p{\descWidth}|}{{\bf Description:}   {\em Fake evolution by setting ID at every step}} \\
\hline{\bf Range} & &  {\bf Default:} (none) \\\multicolumn{1}{|p{\maxVarWidth}|}{\centering *:*} & \multicolumn{2}{p{\paraWidth}|}{anything, 0 as off (default), everything else as on} \\\hline
\end{tabular*}

\vspace{0.5cm}\noindent \begin{tabular*}{\tableWidth}{|c|l@{\extracolsep{\fill}}r|}
\hline
\multicolumn{1}{|p{\maxVarWidth}}{tov\_fast\_interpolation} & {\bf Scope:} private & BOOLEAN \\\hline
\multicolumn{3}{|p{\descWidth}|}{{\bf Description:}   {\em Use faster interpolation algorithm? Default is yes.}} \\
\hline & & {\bf Default:} yes \\\hline
\end{tabular*}

\vspace{0.5cm}\noindent \begin{tabular*}{\tableWidth}{|c|l@{\extracolsep{\fill}}r|}
\hline
\multicolumn{1}{|p{\maxVarWidth}}{tov\_gamma} & {\bf Scope:} private & REAL \\\hline
\multicolumn{3}{|p{\descWidth}|}{{\bf Description:}   {\em The polytropic constant in P = K rho\^Gamma}} \\
\hline{\bf Range} & &  {\bf Default:} 2.0 \\\multicolumn{1}{|p{\maxVarWidth}|}{\centering 1.0:} & \multicolumn{2}{p{\paraWidth}|}{The physical range at high Lorentz factors is [1,2], but otherwise higher values of gamma can also be used} \\\hline
\end{tabular*}

\vspace{0.5cm}\noindent \begin{tabular*}{\tableWidth}{|c|l@{\extracolsep{\fill}}r|}
\hline
\multicolumn{1}{|p{\maxVarWidth}}{tov\_k} & {\bf Scope:} private & REAL \\\hline
\multicolumn{3}{|p{\descWidth}|}{{\bf Description:}   {\em The polytropic constant in P = K rho\^Gamma}} \\
\hline{\bf Range} & &  {\bf Default:} 100.0 \\\multicolumn{1}{|p{\maxVarWidth}|}{\centering (0.0:*} & \multicolumn{2}{p{\paraWidth}|}{Greater than 0} \\\hline
\end{tabular*}

\vspace{0.5cm}\noindent \begin{tabular*}{\tableWidth}{|c|l@{\extracolsep{\fill}}r|}
\hline
\multicolumn{1}{|p{\maxVarWidth}}{tov\_momentum\_psi\_power} & {\bf Scope:} private & INT \\\hline
\multicolumn{3}{|p{\descWidth}|}{{\bf Description:}   {\em Power of Psi to be multiplied with J\^i for Mom}} \\
\hline{\bf Range} & &  {\bf Default:} (none) \\\multicolumn{1}{|p{\maxVarWidth}|}{\centering *:*} & \multicolumn{2}{p{\paraWidth}|}{anything, 0 as default} \\\hline
\end{tabular*}

\vspace{0.5cm}\noindent \begin{tabular*}{\tableWidth}{|c|l@{\extracolsep{\fill}}r|}
\hline
\multicolumn{1}{|p{\maxVarWidth}}{tov\_num\_radial} & {\bf Scope:} private & INT \\\hline
\multicolumn{3}{|p{\descWidth}|}{{\bf Description:}   {\em The number of radial points for the ODE integration}} \\
\hline{\bf Range} & &  {\bf Default:} 100000 \\\multicolumn{1}{|p{\maxVarWidth}|}{\centering 1:*} & \multicolumn{2}{p{\paraWidth}|}{Greater than 0} \\\hline
\end{tabular*}

\vspace{0.5cm}\noindent \begin{tabular*}{\tableWidth}{|c|l@{\extracolsep{\fill}}r|}
\hline
\multicolumn{1}{|p{\maxVarWidth}}{tov\_num\_tovs} & {\bf Scope:} private & INT \\\hline
\multicolumn{3}{|p{\descWidth}|}{{\bf Description:}   {\em The number of TOVs}} \\
\hline{\bf Range} & &  {\bf Default:} 1 \\\multicolumn{1}{|p{\maxVarWidth}|}{\centering 1:*} & \multicolumn{2}{p{\paraWidth}|}{Greater than 0} \\\hline
\end{tabular*}

\vspace{0.5cm}\noindent \begin{tabular*}{\tableWidth}{|c|l@{\extracolsep{\fill}}r|}
\hline
\multicolumn{1}{|p{\maxVarWidth}}{tov\_populate\_timelevels} & {\bf Scope:} private & INT \\\hline
\multicolumn{3}{|p{\descWidth}|}{{\bf Description:}   {\em Populate that amount of timelevels}} \\
\hline{\bf Range} & &  {\bf Default:} 1 \\\multicolumn{1}{|p{\maxVarWidth}|}{\centering 1:3} & \multicolumn{2}{p{\paraWidth}|}{1 (default) to 3} \\\hline
\end{tabular*}

\vspace{0.5cm}\noindent \begin{tabular*}{\tableWidth}{|c|l@{\extracolsep{\fill}}r|}
\hline
\multicolumn{1}{|p{\maxVarWidth}}{tov\_position\_x} & {\bf Scope:} private & REAL \\\hline
\multicolumn{3}{|p{\descWidth}|}{{\bf Description:}   {\em Position of neutron star, x coordinate}} \\
\hline{\bf Range} & &  {\bf Default:} 0.0 \\\multicolumn{1}{|p{\maxVarWidth}|}{\centering *:*} & \multicolumn{2}{p{\paraWidth}|}{real} \\\hline
\end{tabular*}

\vspace{0.5cm}\noindent \begin{tabular*}{\tableWidth}{|c|l@{\extracolsep{\fill}}r|}
\hline
\multicolumn{1}{|p{\maxVarWidth}}{tov\_position\_y} & {\bf Scope:} private & REAL \\\hline
\multicolumn{3}{|p{\descWidth}|}{{\bf Description:}   {\em Position of neutron star, y coordinate}} \\
\hline{\bf Range} & &  {\bf Default:} 0.0 \\\multicolumn{1}{|p{\maxVarWidth}|}{\centering *:*} & \multicolumn{2}{p{\paraWidth}|}{real} \\\hline
\end{tabular*}

\vspace{0.5cm}\noindent \begin{tabular*}{\tableWidth}{|c|l@{\extracolsep{\fill}}r|}
\hline
\multicolumn{1}{|p{\maxVarWidth}}{tov\_position\_z} & {\bf Scope:} private & REAL \\\hline
\multicolumn{3}{|p{\descWidth}|}{{\bf Description:}   {\em Position of neutron star, z coordinate}} \\
\hline{\bf Range} & &  {\bf Default:} 0.0 \\\multicolumn{1}{|p{\maxVarWidth}|}{\centering *:*} & \multicolumn{2}{p{\paraWidth}|}{real} \\\hline
\end{tabular*}

\vspace{0.5cm}\noindent \begin{tabular*}{\tableWidth}{|c|l@{\extracolsep{\fill}}r|}
\hline
\multicolumn{1}{|p{\maxVarWidth}}{tov\_properposition} & {\bf Scope:} private & BOOLEAN \\\hline
\multicolumn{3}{|p{\descWidth}|}{{\bf Description:}   {\em For use only with two NSs, atm only handles equal mass}} \\
\hline & & {\bf Default:} no \\\hline
\end{tabular*}

\vspace{0.5cm}\noindent \begin{tabular*}{\tableWidth}{|c|l@{\extracolsep{\fill}}r|}
\hline
\multicolumn{1}{|p{\maxVarWidth}}{tov\_rho\_central} & {\bf Scope:} private & REAL \\\hline
\multicolumn{3}{|p{\descWidth}|}{{\bf Description:}   {\em The central density}} \\
\hline{\bf Range} & &  {\bf Default:} 1e-3 \\\multicolumn{1}{|p{\maxVarWidth}|}{\centering (0.0:*} & \multicolumn{2}{p{\paraWidth}|}{Must be positive} \\\hline
\end{tabular*}

\vspace{0.5cm}\noindent \begin{tabular*}{\tableWidth}{|c|l@{\extracolsep{\fill}}r|}
\hline
\multicolumn{1}{|p{\maxVarWidth}}{tov\_save\_to\_datafile} & {\bf Scope:} private & STRING \\\hline
\multicolumn{3}{|p{\descWidth}|}{{\bf Description:}   {\em Only save data to file and exit}} \\
\hline{\bf Range} & &  {\bf Default:} (none) \\\multicolumn{1}{|p{\maxVarWidth}|}{\centering .*} & \multicolumn{2}{p{\paraWidth}|}{Any filename, not used if empty} \\\hline
\end{tabular*}

\vspace{0.5cm}\noindent \begin{tabular*}{\tableWidth}{|c|l@{\extracolsep{\fill}}r|}
\hline
\multicolumn{1}{|p{\maxVarWidth}}{tov\_solve\_for\_tovs} & {\bf Scope:} private & INT \\\hline
\multicolumn{3}{|p{\descWidth}|}{{\bf Description:}   {\em Solve for TOVs even if no TOV initial data was requested?}} \\
\hline{\bf Range} & &  {\bf Default:} 3 \\\multicolumn{1}{|p{\maxVarWidth}|}{\centering 0:3} & \multicolumn{2}{p{\paraWidth}|}{"depreciated in favour of TOVSolver::TOV\_Enfor 
ce\_Interpolation"} \\\hline
\end{tabular*}

\vspace{0.5cm}\noindent \begin{tabular*}{\tableWidth}{|c|l@{\extracolsep{\fill}}r|}
\hline
\multicolumn{1}{|p{\maxVarWidth}}{tov\_use\_old\_initial\_data} & {\bf Scope:} private & BOOLEAN \\\hline
\multicolumn{3}{|p{\descWidth}|}{{\bf Description:}   {\em Take old initial data into account (spacetime)? Default is no.}} \\
\hline & & {\bf Default:} no \\\hline
\end{tabular*}

\vspace{0.5cm}\noindent \begin{tabular*}{\tableWidth}{|c|l@{\extracolsep{\fill}}r|}
\hline
\multicolumn{1}{|p{\maxVarWidth}}{tov\_use\_old\_matter\_initial\_data} & {\bf Scope:} private & BOOLEAN \\\hline
\multicolumn{3}{|p{\descWidth}|}{{\bf Description:}   {\em Use also old matter initial data? Default is no.}} \\
\hline & & {\bf Default:} no \\\hline
\end{tabular*}

\vspace{0.5cm}\noindent \begin{tabular*}{\tableWidth}{|c|l@{\extracolsep{\fill}}r|}
\hline
\multicolumn{1}{|p{\maxVarWidth}}{tov\_velocity\_x} & {\bf Scope:} private & REAL \\\hline
\multicolumn{3}{|p{\descWidth}|}{{\bf Description:}   {\em (fixed) Velocity of neutron star, x coordinate (caution!)}} \\
\hline{\bf Range} & &  {\bf Default:} 0.0 \\\multicolumn{1}{|p{\maxVarWidth}|}{\centering *:*} & \multicolumn{2}{p{\paraWidth}|}{real} \\\hline
\end{tabular*}

\vspace{0.5cm}\noindent \begin{tabular*}{\tableWidth}{|c|l@{\extracolsep{\fill}}r|}
\hline
\multicolumn{1}{|p{\maxVarWidth}}{tov\_velocity\_y} & {\bf Scope:} private & REAL \\\hline
\multicolumn{3}{|p{\descWidth}|}{{\bf Description:}   {\em (fixed) Velocity of neutron star, y coordinate (caution!)}} \\
\hline{\bf Range} & &  {\bf Default:} 0.0 \\\multicolumn{1}{|p{\maxVarWidth}|}{\centering *:*} & \multicolumn{2}{p{\paraWidth}|}{real} \\\hline
\end{tabular*}

\vspace{0.5cm}\noindent \begin{tabular*}{\tableWidth}{|c|l@{\extracolsep{\fill}}r|}
\hline
\multicolumn{1}{|p{\maxVarWidth}}{tov\_velocity\_z} & {\bf Scope:} private & REAL \\\hline
\multicolumn{3}{|p{\descWidth}|}{{\bf Description:}   {\em (fixed) Velocity of neutron star, z coordinate (caution!)}} \\
\hline{\bf Range} & &  {\bf Default:} 0.0 \\\multicolumn{1}{|p{\maxVarWidth}|}{\centering *:*} & \multicolumn{2}{p{\paraWidth}|}{real} \\\hline
\end{tabular*}

\vspace{0.5cm}\noindent \begin{tabular*}{\tableWidth}{|c|l@{\extracolsep{\fill}}r|}
\hline
\multicolumn{1}{|p{\maxVarWidth}}{conformal\_storage} & {\bf Scope:} shared from STATICCONFORMAL & KEYWORD \\\hline
\end{tabular*}

\vspace{0.5cm}\parskip = 10pt 

\section{Interfaces} 


\parskip = 0pt

\vspace{3mm} \subsection*{General}

\noindent {\bf Implements}: 

tovsolver
\vspace{2mm}

\noindent {\bf Inherits}: 

admbase

hydrobase

constants

staticconformal
\vspace{2mm}

\vspace{5mm}

\noindent {\bf Uses header}: 

constants.h
\vspace{2mm}

\noindent {\bf Provides}: 



Set\_Rho\_ADM to 

Set\_Momentum\_Source to 

Set\_Initial\_Guess\_for\_u to 

Rescale\_Sources to 
\vspace{2mm}\parskip = 10pt 

\section{Schedule} 


\parskip = 0pt


\noindent This section lists all the variables which are assigned storage by thorn EinsteinInitialData/TOVSolver.  Storage can either last for the duration of the run ({\bf Always} means that if this thorn is activated storage will be assigned, {\bf Conditional} means that if this thorn is activated storage will be assigned for the duration of the run if some condition is met), or can be turned on for the duration of a schedule function.


\subsection*{Storage}NONE
\subsection*{Scheduled Functions}
\vspace{5mm}

\noindent {\bf CCTK\_PARAMCHECK} 

\hspace{5mm} tov\_c\_paramcheck 

\hspace{5mm}{\it check parameters } 


\hspace{5mm}

 \begin{tabular*}{160mm}{cll} 
~ & Language:  & c \\ 
~ & Options:  & global \\ 
~ & Type:  & function \\ 
\end{tabular*} 


\vspace{5mm}

\noindent {\bf CCTK\_WRAGH} 

\hspace{5mm} tov\_c\_allocatememory 

\hspace{5mm}{\it allocate memory for tovsolver\_c } 


\hspace{5mm}

 \begin{tabular*}{160mm}{cll} 
~ & Language:  & c \\ 
~ & Options:  & global \\ 
~ & Type:  & function \\ 
\end{tabular*} 


\vspace{5mm}

\noindent {\bf CCTK\_POSTPOSTINITIAL}   (conditional) 

\hspace{5mm} tov\_c\_freememory 

\hspace{5mm}{\it free memory from tovsolver\_c } 


\hspace{5mm}

 \begin{tabular*}{160mm}{cll} 
~ & Language:  & c \\ 
~ & Options:  & global \\ 
~ & Type:  & function \\ 
\end{tabular*} 


\vspace{5mm}

\noindent {\bf HydroBase\_Initial}   (conditional) 

\hspace{5mm} tov\_initial\_data 

\hspace{5mm}{\it group for the tov initial data } 


\hspace{5mm}

 \begin{tabular*}{160mm}{cll} 
~ & Sync:  & admbase::metric \\ 
~& ~ &admbase::curv\\ 
~& ~ &admbase::lapse\\ 
~& ~ &admbase::shift\\ 
~& ~ &rho\\ 
~& ~ &press\\ 
~& ~ &eps\\ 
~& ~ &vel\\ 
~& ~ &w\_lorentz\\ 
~ & Type:  & group \\ 
\end{tabular*} 


\vspace{5mm}

\noindent {\bf TOV\_Initial\_Data}   (conditional) 

\hspace{5mm} tov\_c\_integrate\_rhs 

\hspace{5mm}{\it integrate the 1d equations for the tov star } 


\hspace{5mm}

 \begin{tabular*}{160mm}{cll} 
~ & Language:  & c \\ 
~ & Options:  & global \\ 
~ & Type:  & function \\ 
~ & Writes:  & staticconformal::conformal\_state(everywhere) \\ 
\end{tabular*} 


\vspace{5mm}

\noindent {\bf TOV\_Initial\_Data}   (conditional) 

\hspace{5mm} tov\_write\_1d\_datafile 

\hspace{5mm}{\it save data to file and exit } 


\hspace{5mm}

 \begin{tabular*}{160mm}{cll} 
~ & After:  & tov\_c\_integrate\_rhs \\ 
~ & Before:  & tov\_c\_exact \\ 
~ & Language:  & c \\ 
~ & Options:  & global \\ 
~ & Type:  & function \\ 
\end{tabular*} 


\vspace{5mm}

\noindent {\bf TOV\_Initial\_Data}   (conditional) 

\hspace{5mm} tov\_set\_properpositions 

\hspace{5mm}{\it steer ns position parameters according to proper distance } 


\hspace{5mm}

 \begin{tabular*}{160mm}{cll} 
~ & After:  & tov\_c\_integrate\_rhs \\ 
~ & Before:  & tov\_c\_exact \\ 
~ & Language:  & c \\ 
~ & Options:  & global \\ 
~ & Type:  & function \\ 
\end{tabular*} 


\vspace{5mm}

\noindent {\bf TOV\_Initial\_Data}   (conditional) 

\hspace{5mm} tov\_c\_exact 

\hspace{5mm}{\it set up the 3d quantities for the tov star } 


\hspace{5mm}

 \begin{tabular*}{160mm}{cll} 
~ & After:  & tov\_c\_integrate\_rhs \\ 
~ & Language:  & c \\ 
~ & Reads:  & staticconformal::conformal\_state \\ 
~& ~ &admbase::shift\_state\\ 
~& ~ &grid::coordinates\\ 
~ & Type:  & function \\ 
~ & Writes:  & admbase::alp(everywhere) \\ 
~& ~ &admbase::curv(everywhere)\\ 
~& ~ &admbase::metric\_p\_p(everywhere)\\ 
~& ~ &admbase::metric\_p(everywhere)\\ 
~& ~ &admbase::metric(everywhere)\\ 
~& ~ &admbase::shift(everywhere)\\ 
~& ~ &hydrobase::eps\_p\_p(everywhere)\\ 
~& ~ &hydrobase::rho\_p\_p(everywhere)\\ 
~& ~ &hydrobase::vel\_p\_p(everywhere)\\ 
~& ~ &hydrobase::w\_lorentz\_p\_p(everywhere)\\ 
~& ~ &hydrobase::eps\_p(everywhere)\\ 
~& ~ &hydrobase::rho\_p(everywhere)\\ 
~& ~ &hydrobase::vel\_p(everywhere)\\ 
~& ~ &hydrobase::w\_lorentz\_p(everywhere)\\ 
~& ~ &hydrobase::eps(everywhere)\\ 
~& ~ &hydrobase::rho(everywhere)\\ 
~& ~ &hydrobase::vel(everywhere)\\ 
~& ~ &hydrobase::w\_lorentz(everywhere)\\ 
~& ~ &hydrobase::press(everywhere)\\ 
~& ~ &staticconformal::psi(everywhere)\\ 
~& ~ &staticconformal::confac\_1derivs(everywhere)\\ 
~& ~ &staticconformal::confac\_2derivs(everywhere)\\ 
\end{tabular*} 


\vspace{5mm}

\noindent {\bf }   (conditional) 

\hspace{5mm} tov\_prepare\_fake\_evolution 

\hspace{5mm}{\it prepare for fake evolution } 


\hspace{5mm}

 \begin{tabular*}{160mm}{cll} 
~ & After:  & tov\_c\_exact \\ 
~ & Language:  & c \\ 
~ & Type:  & function \\ 
\end{tabular*} 


\vspace{5mm}

\noindent {\bf MoL\_PostStep}   (conditional) 

\hspace{5mm} tov\_c\_exact 

\hspace{5mm}{\it use fake evolution } 


\hspace{5mm}

 \begin{tabular*}{160mm}{cll} 
~ & After:  & hydrobase\_poststep \\ 
~ & Language:  & c \\ 
~ & Reads:  & staticconformal::conformal\_state \\ 
~& ~ &admbase::shift\_state\\ 
~ & Type:  & function \\ 
~ & Writes:  & admbase::alp(everywhere) \\ 
~& ~ &admbase::curv(everywhere)\\ 
~& ~ &admbase::metric(everywhere)\\ 
~& ~ &admbase::shift(everywhere)\\ 
~& ~ &hydrobase::eps(everywhere)\\ 
~& ~ &hydrobase::press(everywhere)\\ 
~& ~ &hydrobase::rho(everywhere)\\ 
~& ~ &hydrobase::w\_lorentz(everywhere)\\ 
~& ~ &hydrobase::vel(everywhere)\\ 
~& ~ &staticconformal::psi(everywhere)\\ 
~& ~ &staticconformal::confac\_1derivs(everywhere)\\ 
~& ~ &staticconformal::confac\_2derivs(everywhere)\\ 
\end{tabular*} 



\vspace{5mm}\parskip = 10pt 
\end{document}
