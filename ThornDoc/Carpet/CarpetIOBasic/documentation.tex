% *======================================================================*
%  Cactus Thorn template for ThornGuide documentation
%  Author: Ian Kelley
%  Date: Sun Jun 02, 2002
%  $Header: /cactusdevcvs/Cactus/doc/ThornGuide/template.tex,v 1.12 2004/01/07 20:12:39 rideout Exp $
%
%  Thorn documentation in the latex file doc/documentation.tex
%  will be included in ThornGuides built with the Cactus make system.
%  The scripts employed by the make system automatically include
%  pages about variables, parameters and scheduling parsed from the
%  relevant thorn CCL files.
%
%  This template contains guidelines which help to assure that your
%  documentation will be correctly added to ThornGuides. More
%  information is available in the Cactus UsersGuide.
%
%  Guidelines:
%   - Do not change anything before the line
%       % START CACTUS THORNGUIDE",
%     except for filling in the title, author, date, etc. fields.
%        - Each of these fields should only be on ONE line.
%        - Author names should be separated with a \\ or a comma.
%   - You can define your own macros, but they must appear after
%     the START CACTUS THORNGUIDE line, and must not redefine standard
%     latex commands.
%   - To avoid name clashes with other thorns, 'labels', 'citations',
%     'references', and 'image' names should conform to the following
%     convention:
%       ARRANGEMENT_THORN_LABEL
%     For example, an image wave.eps in the arrangement CactusWave and
%     thorn WaveToyC should be renamed to CactusWave_WaveToyC_wave.eps
%   - Graphics should only be included using the graphicx package.
%     More specifically, with the "\includegraphics" command.  Do
%     not specify any graphic file extensions in your .tex file. This
%     will allow us to create a PDF version of the ThornGuide
%     via pdflatex.
%   - References should be included with the latex "\bibitem" command.
%   - Use \begin{abstract}...\end{abstract} instead of \abstract{...}
%   - Do not use \appendix, instead include any appendices you need as
%     standard sections.
%   - For the benefit of our Perl scripts, and for future extensions,
%     please use simple latex.
%
% *======================================================================*
%
% Example of including a graphic image:
%    \begin{figure}[ht]
% 	\begin{center}
%    	   \includegraphics[width=6cm]{/home/runner/work/einsteintoolkit/einsteintoolkit/arrangements/Carpet/CarpetIOBasic/doc/MyArrangement_MyThorn_MyFigure}
% 	\end{center}
% 	\caption{Illustration of this and that}
% 	\label{MyArrangement_MyThorn_MyLabel}
%    \end{figure}
%
% Example of using a label:
%   \label{MyArrangement_MyThorn_MyLabel}
%
% Example of a citation:
%    \cite{MyArrangement_MyThorn_Author99}
%
% Example of including a reference
%   \bibitem{MyArrangement_MyThorn_Author99}
%   {J. Author, {\em The Title of the Book, Journal, or periodical}, 1 (1999),
%   1--16. {\tt http://www.nowhere.com/}}
%
% *======================================================================*

% If you are using CVS use this line to give version information
% $Header: /cactusdevcvs/Cactus/doc/ThornGuide/template.tex,v 1.12 2004/01/07 20:12:39 rideout Exp $

\documentclass{article}

% Use the Cactus ThornGuide style file
% (Automatically used from Cactus distribution, if you have a
%  thorn without the Cactus Flesh download this from the Cactus
%  homepage at www.cactuscode.org)
\usepackage{../../../../../doc/latex/cactus}

\newlength{\tableWidth} \newlength{\maxVarWidth} \newlength{\paraWidth} \newlength{\descWidth} \begin{document}

% The author of the documentation
\author{Erik Schnetter \textless schnetter@cct.lsu.edu\textgreater}

% The title of the document (not necessarily the name of the Thorn)
\title{CarpetIOBasic}

% the date your document was last changed, if your document is in CVS,
% please use:
%    \date{$ $Date: 2004/01/07 20:12:39 $ $}
\date{December 20 2005}

\maketitle

% Do not delete next line
% START CACTUS THORNGUIDE

% Add all definitions used in this documentation here
%   \def\mydef etc

% Add an abstract for this thorn's documentation
\begin{abstract}

\end{abstract}

% The following sections are suggestive only.
% Remove them or add your own.

\section{Introduction}

\section{Physical System}

\section{Numerical Implementation}

\section{Using This Thorn}

\subsection{Obtaining This Thorn}

\subsection{Basic Usage}

\subsection{Special Behaviour}

\subsection{Interaction With Other Thorns}

\subsection{Examples}

\subsection{Support and Feedback}

\section{History}

\subsection{Thorn Source Code}

\subsection{Thorn Documentation}

\subsection{Acknowledgements}


\begin{thebibliography}{9}

\end{thebibliography}

% Do not delete next line
% END CACTUS THORNGUIDE



\section{Parameters} 


\parskip = 0pt

\setlength{\tableWidth}{160mm}

\setlength{\paraWidth}{\tableWidth}
\setlength{\descWidth}{\tableWidth}
\settowidth{\maxVarWidth}{strict\_io\_parameter\_check}

\addtolength{\paraWidth}{-\maxVarWidth}
\addtolength{\paraWidth}{-\columnsep}
\addtolength{\paraWidth}{-\columnsep}
\addtolength{\paraWidth}{-\columnsep}

\addtolength{\descWidth}{-\columnsep}
\addtolength{\descWidth}{-\columnsep}
\addtolength{\descWidth}{-\columnsep}
\noindent \begin{tabular*}{\tableWidth}{|c|l@{\extracolsep{\fill}}r|}
\hline
\multicolumn{1}{|p{\maxVarWidth}}{int\_width} & {\bf Scope:} private & INT \\\hline
\multicolumn{3}{|p{\descWidth}|}{{\bf Description:}   {\em Field width for integer values}} \\
\hline{\bf Range} & &  {\bf Default:} 9 \\\multicolumn{1}{|p{\maxVarWidth}|}{\centering 1:*} & \multicolumn{2}{p{\paraWidth}|}{} \\\hline
\end{tabular*}

\vspace{0.5cm}\noindent \begin{tabular*}{\tableWidth}{|c|l@{\extracolsep{\fill}}r|}
\hline
\multicolumn{1}{|p{\maxVarWidth}}{iter\_width} & {\bf Scope:} private & INT \\\hline
\multicolumn{3}{|p{\descWidth}|}{{\bf Description:}   {\em Field width for the current iteration}} \\
\hline{\bf Range} & &  {\bf Default:} 9 \\\multicolumn{1}{|p{\maxVarWidth}|}{\centering 1:*} & \multicolumn{2}{p{\paraWidth}|}{} \\\hline
\end{tabular*}

\vspace{0.5cm}\noindent \begin{tabular*}{\tableWidth}{|c|l@{\extracolsep{\fill}}r|}
\hline
\multicolumn{1}{|p{\maxVarWidth}}{outheader\_every} & {\bf Scope:} private & INT \\\hline
\multicolumn{3}{|p{\descWidth}|}{{\bf Description:}   {\em How often to print the header}} \\
\hline{\bf Range} & &  {\bf Default:} 20 \\\multicolumn{1}{|p{\maxVarWidth}|}{\centering 1:*} & \multicolumn{2}{p{\paraWidth}|}{Output every so many time steps} \\\multicolumn{1}{|p{\maxVarWidth}|}{\centering -1} & \multicolumn{2}{p{\paraWidth}|}{No header output} \\\hline
\end{tabular*}

\vspace{0.5cm}\noindent \begin{tabular*}{\tableWidth}{|c|l@{\extracolsep{\fill}}r|}
\hline
\multicolumn{1}{|p{\maxVarWidth}}{outinfo\_criterion} & {\bf Scope:} private & KEYWORD \\\hline
\multicolumn{3}{|p{\descWidth}|}{{\bf Description:}   {\em Criterion to select scalar output intervals, overrides out\_every}} \\
\hline{\bf Range} & &  {\bf Default:} iteration \\\multicolumn{1}{|p{\maxVarWidth}|}{\centering never} & \multicolumn{2}{p{\paraWidth}|}{Never output} \\\multicolumn{1}{|p{\maxVarWidth}|}{\centering iteration} & \multicolumn{2}{p{\paraWidth}|}{Output every so many iterations} \\\multicolumn{1}{|p{\maxVarWidth}|}{\centering divisor} & \multicolumn{2}{p{\paraWidth}|}{Output if iteration mod divisor == 0.} \\\multicolumn{1}{|p{\maxVarWidth}|}{\centering time} & \multicolumn{2}{p{\paraWidth}|}{Output every that much coordinate time} \\\hline
\end{tabular*}

\vspace{0.5cm}\noindent \begin{tabular*}{\tableWidth}{|c|l@{\extracolsep{\fill}}r|}
\hline
\multicolumn{1}{|p{\maxVarWidth}}{outinfo\_dt} & {\bf Scope:} private & REAL \\\hline
\multicolumn{3}{|p{\descWidth}|}{{\bf Description:}   {\em How often to do scalar output, overrides out\_dt}} \\
\hline{\bf Range} & &  {\bf Default:} -2 \\\multicolumn{1}{|p{\maxVarWidth}|}{\centering (0:*} & \multicolumn{2}{p{\paraWidth}|}{In intervals of that much coordinate time} \\\multicolumn{1}{|p{\maxVarWidth}|}{\centering } & \multicolumn{2}{p{\paraWidth}|}{As often as possible} \\\multicolumn{1}{|p{\maxVarWidth}|}{\centering -1} & \multicolumn{2}{p{\paraWidth}|}{Disable output} \\\multicolumn{1}{|p{\maxVarWidth}|}{\centering -2} & \multicolumn{2}{p{\paraWidth}|}{Default to IO::out\_dt} \\\hline
\end{tabular*}

\vspace{0.5cm}\noindent \begin{tabular*}{\tableWidth}{|c|l@{\extracolsep{\fill}}r|}
\hline
\multicolumn{1}{|p{\maxVarWidth}}{outinfo\_every} & {\bf Scope:} private & INT \\\hline
\multicolumn{3}{|p{\descWidth}|}{{\bf Description:}   {\em How often to do scalar output, overrides IO::out\_every}} \\
\hline{\bf Range} & &  {\bf Default:} -2 \\\multicolumn{1}{|p{\maxVarWidth}|}{\centering 1:*} & \multicolumn{2}{p{\paraWidth}|}{Output every so many time steps} \\\multicolumn{1}{|p{\maxVarWidth}|}{\centering -1:0} & \multicolumn{2}{p{\paraWidth}|}{No output} \\\multicolumn{1}{|p{\maxVarWidth}|}{\centering -2} & \multicolumn{2}{p{\paraWidth}|}{Default to IO::out\_every} \\\hline
\end{tabular*}

\vspace{0.5cm}\noindent \begin{tabular*}{\tableWidth}{|c|l@{\extracolsep{\fill}}r|}
\hline
\multicolumn{1}{|p{\maxVarWidth}}{outinfo\_reductions} & {\bf Scope:} private & STRING \\\hline
\multicolumn{3}{|p{\descWidth}|}{{\bf Description:}   {\em List of reductions to output in scalar form}} \\
\hline{\bf Range} & &  {\bf Default:} minimum maximum \\\multicolumn{1}{|p{\maxVarWidth}|}{\centering } & \multicolumn{2}{p{\paraWidth}|}{A regex which matches everything} \\\hline
\end{tabular*}

\vspace{0.5cm}\noindent \begin{tabular*}{\tableWidth}{|c|l@{\extracolsep{\fill}}r|}
\hline
\multicolumn{1}{|p{\maxVarWidth}}{outinfo\_vars} & {\bf Scope:} private & STRING \\\hline
\multicolumn{3}{|p{\descWidth}|}{{\bf Description:}   {\em Variables to output in scalar form}} \\
\hline{\bf Range} & &  {\bf Default:} (none) \\\multicolumn{1}{|p{\maxVarWidth}|}{\centering } & \multicolumn{2}{p{\paraWidth}|}{A regex which matches everything} \\\hline
\end{tabular*}

\vspace{0.5cm}\noindent \begin{tabular*}{\tableWidth}{|c|l@{\extracolsep{\fill}}r|}
\hline
\multicolumn{1}{|p{\maxVarWidth}}{real\_max} & {\bf Scope:} private & REAL \\\hline
\multicolumn{3}{|p{\descWidth}|}{{\bf Description:}   {\em Upper bound for numbers that are displayed in fixed notation}} \\
\hline{\bf Range} & &  {\bf Default:} 1.0e+3 \\\multicolumn{1}{|p{\maxVarWidth}|}{\centering (0.0:*} & \multicolumn{2}{p{\paraWidth}|}{} \\\hline
\end{tabular*}

\vspace{0.5cm}\noindent \begin{tabular*}{\tableWidth}{|c|l@{\extracolsep{\fill}}r|}
\hline
\multicolumn{1}{|p{\maxVarWidth}}{real\_min} & {\bf Scope:} private & REAL \\\hline
\multicolumn{3}{|p{\descWidth}|}{{\bf Description:}   {\em Lower bound for numbers that are displayed in fixed notation}} \\
\hline{\bf Range} & &  {\bf Default:} 1.0e-8 \\\multicolumn{1}{|p{\maxVarWidth}|}{\centering (0.0:*} & \multicolumn{2}{p{\paraWidth}|}{} \\\hline
\end{tabular*}

\vspace{0.5cm}\noindent \begin{tabular*}{\tableWidth}{|c|l@{\extracolsep{\fill}}r|}
\hline
\multicolumn{1}{|p{\maxVarWidth}}{real\_prec} & {\bf Scope:} private & INT \\\hline
\multicolumn{3}{|p{\descWidth}|}{{\bf Description:}   {\em Precision for real values}} \\
\hline{\bf Range} & &  {\bf Default:} 7 \\\multicolumn{1}{|p{\maxVarWidth}|}{\centering 0:*} & \multicolumn{2}{p{\paraWidth}|}{} \\\hline
\end{tabular*}

\vspace{0.5cm}\noindent \begin{tabular*}{\tableWidth}{|c|l@{\extracolsep{\fill}}r|}
\hline
\multicolumn{1}{|p{\maxVarWidth}}{real\_prec\_sci} & {\bf Scope:} private & INT \\\hline
\multicolumn{3}{|p{\descWidth}|}{{\bf Description:}   {\em Precision for real values in scientific notation}} \\
\hline{\bf Range} & &  {\bf Default:} 6 \\\multicolumn{1}{|p{\maxVarWidth}|}{\centering 0:*} & \multicolumn{2}{p{\paraWidth}|}{} \\\hline
\end{tabular*}

\vspace{0.5cm}\noindent \begin{tabular*}{\tableWidth}{|c|l@{\extracolsep{\fill}}r|}
\hline
\multicolumn{1}{|p{\maxVarWidth}}{real\_width} & {\bf Scope:} private & INT \\\hline
\multicolumn{3}{|p{\descWidth}|}{{\bf Description:}   {\em Field width for real values}} \\
\hline{\bf Range} & &  {\bf Default:} 12 \\\multicolumn{1}{|p{\maxVarWidth}|}{\centering 1:*} & \multicolumn{2}{p{\paraWidth}|}{} \\\hline
\end{tabular*}

\vspace{0.5cm}\noindent \begin{tabular*}{\tableWidth}{|c|l@{\extracolsep{\fill}}r|}
\hline
\multicolumn{1}{|p{\maxVarWidth}}{time\_prec} & {\bf Scope:} private & INT \\\hline
\multicolumn{3}{|p{\descWidth}|}{{\bf Description:}   {\em Precision for the simulation time}} \\
\hline{\bf Range} & &  {\bf Default:} 3 \\\multicolumn{1}{|p{\maxVarWidth}|}{\centering 0:*} & \multicolumn{2}{p{\paraWidth}|}{} \\\hline
\end{tabular*}

\vspace{0.5cm}\noindent \begin{tabular*}{\tableWidth}{|c|l@{\extracolsep{\fill}}r|}
\hline
\multicolumn{1}{|p{\maxVarWidth}}{time\_width} & {\bf Scope:} private & INT \\\hline
\multicolumn{3}{|p{\descWidth}|}{{\bf Description:}   {\em Field width for the simulation time}} \\
\hline{\bf Range} & &  {\bf Default:} 9 \\\multicolumn{1}{|p{\maxVarWidth}|}{\centering 1:*} & \multicolumn{2}{p{\paraWidth}|}{} \\\hline
\end{tabular*}

\vspace{0.5cm}\noindent \begin{tabular*}{\tableWidth}{|c|l@{\extracolsep{\fill}}r|}
\hline
\multicolumn{1}{|p{\maxVarWidth}}{out\_criterion} & {\bf Scope:} shared from IO & KEYWORD \\\hline
\end{tabular*}

\vspace{0.5cm}\noindent \begin{tabular*}{\tableWidth}{|c|l@{\extracolsep{\fill}}r|}
\hline
\multicolumn{1}{|p{\maxVarWidth}}{out\_dt} & {\bf Scope:} shared from IO & REAL \\\hline
\end{tabular*}

\vspace{0.5cm}\noindent \begin{tabular*}{\tableWidth}{|c|l@{\extracolsep{\fill}}r|}
\hline
\multicolumn{1}{|p{\maxVarWidth}}{out\_every} & {\bf Scope:} shared from IO & INT \\\hline
\end{tabular*}

\vspace{0.5cm}\noindent \begin{tabular*}{\tableWidth}{|c|l@{\extracolsep{\fill}}r|}
\hline
\multicolumn{1}{|p{\maxVarWidth}}{strict\_io\_parameter\_check} & {\bf Scope:} shared from IO & BOOLEAN \\\hline
\end{tabular*}

\vspace{0.5cm}\parskip = 10pt 

\section{Interfaces} 


\parskip = 0pt

\vspace{3mm} \subsection*{General}

\noindent {\bf Implements}: 

iobasic
\vspace{2mm}

\noindent {\bf Inherits}: 

io
\vspace{2mm}
\subsection*{Grid Variables}
\vspace{5mm}\subsubsection{PRIVATE GROUPS}

\vspace{5mm}

\begin{tabular*}{150mm}{|c|c@{\extracolsep{\fill}}|rl|} \hline 
~ {\bf Group Names} ~ & ~ {\bf Variable Names} ~  &{\bf Details} ~ & ~\\ 
\hline 
last\_output\_iteration & last\_output\_iteration & compact & 0 \\ 
 &  & dimensions & 0 \\ 
 &  & distribution & CONSTANT \\ 
 &  & group type & SCALAR \\ 
 &  & timelevels & 1 \\ 
 &  & variable type & INT \\ 
\hline 
last\_output\_time & last\_output\_time & compact & 0 \\ 
 &  & dimensions & 0 \\ 
 &  & distribution & CONSTANT \\ 
 &  & group type & SCALAR \\ 
 &  & timelevels & 1 \\ 
 &  & variable type & REAL \\ 
\hline 
this\_iteration & this\_iteration & compact & 0 \\ 
 &  & dimensions & 0 \\ 
 &  & distribution & CONSTANT \\ 
 &  & group type & SCALAR \\ 
 &  & timelevels & 1 \\ 
 &  & variable type & INT \\ 
\hline 
\end{tabular*} 



\vspace{5mm}

\noindent {\bf Uses header}: 

Timer.hh

carpet.hh

typecase.hh

typeprops.hh
\vspace{2mm}\parskip = 10pt 

\section{Schedule} 


\parskip = 0pt


\noindent This section lists all the variables which are assigned storage by thorn Carpet/CarpetIOBasic.  Storage can either last for the duration of the run ({\bf Always} means that if this thorn is activated storage will be assigned, {\bf Conditional} means that if this thorn is activated storage will be assigned for the duration of the run if some condition is met), or can be turned on for the duration of a schedule function.


\subsection*{Storage}

\hspace{5mm}

 \begin{tabular*}{160mm}{ll} 

{\bf Always:}&  ~ \\ 
 last\_output\_iteration last\_output\_time this\_iteration & ~\\ 
~ & ~\\ 
\end{tabular*} 


\subsection*{Scheduled Functions}
\vspace{5mm}

\noindent {\bf CCTK\_STARTUP} 

\hspace{5mm} carpetiobasicstartup 

\hspace{5mm}{\it startup routine } 


\hspace{5mm}

 \begin{tabular*}{160mm}{cll} 
~ & After:  & ioutil\_startup \\ 
~ & Language:  & c \\ 
~ & Options:  & global \\ 
~ & Type:  & function \\ 
\end{tabular*} 


\vspace{5mm}

\noindent {\bf CCTK\_INITIAL} 

\hspace{5mm} carpetiobasicinit 

\hspace{5mm}{\it initialisation routine } 


\hspace{5mm}

 \begin{tabular*}{160mm}{cll} 
~ & Language:  & c \\ 
~ & Options:  & global \\ 
~ & Type:  & function \\ 
~ & Writes:  & carpetiobasic::last\_output\_iteration \\ 
~& ~ &carpetiobasic::last\_output\_time\\ 
~& ~ &carpetiobasic::this\_iteration\\ 
\end{tabular*} 



\vspace{5mm}\parskip = 10pt 
\end{document}
