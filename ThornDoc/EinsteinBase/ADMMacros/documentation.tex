% *======================================================================*
%  Cactus Thorn template for ThornGuide documentation
%  Author: Ian Kelley
%  Date: Sun Jun 02, 2002
%  $Header$
%
%  Thorn documentation in the latex file doc/documentation.tex
%  will be included in ThornGuides built with the Cactus make system.
%  The scripts employed by the make system automatically include
%  pages about variables, parameters and scheduling parsed from the
%  relevant thorn CCL files.
%
%  This template contains guidelines which help to assure that your
%  documentation will be correctly added to ThornGuides. More
%  information is available in the Cactus UsersGuide.
%
%  Guidelines:
%   - Do not change anything before the line
%       % START CACTUS THORNGUIDE",
%     except for filling in the title, author, date, etc. fields.
%        - Each of these fields should only be on ONE line.
%        - Author names should be separated with a \\ or a comma.
%   - You can define your own macros, but they must appear after
%     the START CACTUS THORNGUIDE line, and must not redefine standard
%     latex commands.
%   - To avoid name clashes with other thorns, 'labels', 'citations',
%     'references', and 'image' names should conform to the following
%     convention:
%       ARRANGEMENT_THORN_LABEL
%     For example, an image wave.eps in the arrangement CactusWave and
%     thorn WaveToyC should be renamed to CactusWave_WaveToyC_wave.eps
%   - Graphics should only be included using the graphicx package.
%     More specifically, with the "\includegraphics" command.  Do
%     not specify any graphic file extensions in your .tex file. This
%     will allow us to create a PDF version of the ThornGuide
%     via pdflatex.
%   - References should be included with the latex "\bibitem" command.
%   - Use \begin{abstract}...\end{abstract} instead of \abstract{...}
%   - Do not use \appendix, instead include any appendices you need as
%     standard sections.
%   - For the benefit of our Perl scripts, and for future extensions,
%     please use simple latex.
%
% *======================================================================*
%
% Example of including a graphic image:
%    \begin{figure}[ht]
% 	\begin{center}
%    	   \includegraphics[width=6cm]{/home/runner/work/einsteintoolkit/einsteintoolkit/arrangements/EinsteinBase/ADMMacros/doc/MyArrangement_MyThorn_MyFigure}
% 	\end{center}
% 	\caption{Illustration of this and that}
% 	\label{MyArrangement_MyThorn_MyLabel}
%    \end{figure}
%
% Example of using a label:
%   \label{MyArrangement_MyThorn_MyLabel}
%
% Example of a citation:
%    \cite{MyArrangement_MyThorn_Author99}
%
% Example of including a reference
%   \bibitem{MyArrangement_MyThorn_Author99}
%   {J. Author, {\em The Title of the Book, Journal, or periodical}, 1 (1999),
%   1--16. \texttt{http://www.nowhere.com/}}
%
% *======================================================================*

% If you are using CVS use this line to give version information
% $Header$

\documentclass{article}

% Use the Cactus ThornGuide style file
% (Automatically used from Cactus distribution, if you have a
%  thorn without the Cactus Flesh download this from the Cactus
%  homepage at www.cactuscode.org)
\usepackage{../../../../../doc/latex/cactus}

\newlength{\tableWidth} \newlength{\maxVarWidth} \newlength{\paraWidth} \newlength{\descWidth} \begin{document}

\title{ADMMacros}
\author{Original code by Gabrielle Allen,\\
	later enhancements by Denis Pollney}
\date{$ $Date$ $}

\maketitle

% Do not delete next line
% START CACTUS THORNGUIDE

% Add all definitions used in this documentation here
%   \def\mydef etc
\def\ie{i.e.\hbox{}}
\def\del{\nabla}

% get size/spacing of "++" right, cf online C++ FAQ question 35.1
\def\Cplusplus{\hbox{C\raise.25ex\hbox{\footnotesize ++}}}


\begin{abstract}
Provides macros for common relativity calculations, using the
\textbf{ADMBase} variables.
\end{abstract}


\section{Purpose}

This thorn provides various macros which can be used to calculate
quantities, such as the Christoffel Symbol or Riemann Tensor
components, using the basic variables of thorn \textbf{ADMBase}.
The macros can be used in both
Fortran and C code.  The macros work pointwise to calculate quantities
at the grid point $(\code{i},\code{j},\code{k})$; it's up to you
to loop over all the grid points where you want computations done.
The macros are written in such a way that within any single loop,
quantities which have already been calculated are automagically
reused as needed in later calculations.

\subsection{Finite Differencing}

By default, the macros use centered 2nd~order finite differencing,
with 3-point finite difference molecules.
That is, when finite differencing the the grid-point indices
$\code{i} \pm 1$, $\code{j} \pm 1$, and $\code{k} \pm 1$
must also be valid, and \code{driver::ghost\_size} must be set
to at least $1$.

Some of the macros also support centered 4th~order finite differencing;
This is selected with the parameter \verb|spatial_order|.  This may be
set to either~$2$ or~$4$; it defaults to~$2$.  If it's set to~$4$, then
5-point finite difference molecules are used, so the grid-point indices
$\code{i} \pm 2$, $\code{j} \pm 2$, and $\code{k} \pm 2$
must also be valid, and \code{driver::ghost\_size} must be set
to at least $2$.  The only save way to be certain which macros support
4th~order finite differencing is to check the source code; the macros
which don't support it simply hard-code 2nd~order finite differencing
and ignore the \code{spatial\_order} parameter.

At present 4th~order finite differencing is only supported for Fortran
code.  (That is, at present the C~versions of the macros all ignore the
\code{spatial\_order} parameter.)


\section{Using ADM Macros}

Each macro described in Section~\ref{admmacros:macros} is implemented
using three include files:
\begin{description}
\item[\texttt{<MACRONAME>\_declare.h}]
        sets up the declarations for the internal macro variables.
        All the internal (hidden) variables have names beginning
        with the macro name. This file should be included in the
        declarations section of your routine.

\item[\texttt{<MACRONAME>\_guts.h}]
	is the actual included source code which will calculate the quantities.

\item[\texttt{<MACRONAME>\_undefine.h}]
	resets the macros. This file \textbf{must} be \code{\#include}d
	at the end of every loop using macros.  Without this, a second loop
        using macros would assume that quantities have already been calculated
	(and thus get wrong results).
\end{description}

The macros which compute derivatives also use the following variables;
you should avoid using these in your code, in either lower or upper case:
\begin{verbatim}
di2, dj2, dk2        /* only in C, not in Fortran */
dt, dx, dy, dz
idx, idy, idz
i2dx, i2dy, i2dz
i12dx, i12dy, i12dz
idxx, idxy, idxz, idyy, idyz, idzz
i12dxx, i12dyy, i12dzz
i36dxy, i36dxz, i36dyz
\end{verbatim}

To use the macros, first find the name of the macro from the table in
Section~\ref{admmacros:macros} and put the include files in the
correct place following the instructions above. Note that all ADMMacro
include files are in the directory \texttt{EinsteinBase/ADMMacros/src/macro/},
so you include the macros with lines such as
\begin{verbatim}
#include "EinsteinBase/ADMMacros/src/macro/<MACRONAME>_<TYPE>.h"
\end{verbatim}
(Recall that Cactus uses a C-style preprocessor for Fortran as well as
C/\Cplusplus{} code; you use the same \code{\#include}s for all these
languages.)

Each variable that the macro calculates is listed in the table of
Section~\ref{admmacros:macros}.  Note that these variable names are
themselves macros and are case sensitive. \textbf{Always use the macro
variables on the right hand sides of equations, never redefine them
yourself, since they may be used in later (hidden) calculations.}

\subsection{Fortran}

If you are using the macros inside a Fortran function then the
\code{i}, \code{j} and \code{k} indices are used directly.

If you're using (either directly or indirectly) any macro which computes
derivatives, you also need to \code{\#include} two additional files:
\begin{description}
\item[\texttt{ADM\_Spacing\_declare.h}]\mbox{}\\
	This must be \code{\#include}d \textbf{before} any of the
	other \texttt{<MACRONAME>\_declare.h} files.
\item[\texttt{ADM\_Spacing.h}]\mbox{}\\
	This must be \code{\#include}d \textbf{after} all of the
	other \texttt{<MACRONAME>\_declare.h} and \textbf{before}
	any of the \texttt{<MACRONAME>\_guts.h} files.
\end{description}

The Fortran example below should make this clear(er).

\subsection{C}

If you are using the macros inside a C function then you must define
the grid-function subscripting index \code{ijk}, which can be found
from \code{i}, \code{j} and \code{k} using the macro
\code{CCTK\_GFINDEX3D(cctkGH,i,j,k)}.  Of course, since \code{ijk}
depends on \code{i}, \code{j} and \code{k}, you have to assign
\code{ijk} its value \textbf{inside} the loop-over-grid-points loops.

You must also define the grid-function strides \code{di}, \code{dj}
and \code{dk} to give the grid-function subscripting index offsets of
the grid points $(i+1,j,k)$, $(i,j+1,k)$, and $(i,j,k+1)$ (respectively)
relative to $(i,j,k)$.  That is, you should define \code{di = 1},
\code{dj = cctk\_lsh[0]}, and \code{dk = cctk\_lsh[0]*cctk\_lsh[1]}.
Since these don't depend on \code{i}, \code{j} and \code{k},
they can be assigned values once outside the loop-over-grid-points loops.

The C example below should make this clear(er).

Note that you should assign all these variables their values \textbf{before}
\code{\#include}ing the \texttt{<MACRONAME>\_guts.h} macro (it may do
calculations which use these values).


\section{Examples}

\subsection{Fortran}

This example comes from thorn \texttt{CactusArchive/Maximal} and uses
the $trK$ macro to calculate the trace of the extrinsic curvature.

\newpage	% we'd like the entire example to fit on a single page
\begin{verbatim}
c     Declarations for macros.
#include "EinsteinBase/ADMMacros/src/macro/TRK_declare.h"

c we're not taking any derivatives here, but if we were,
c we would also need the following line:
#include "EinsteinBase/ADMMacros/src/macro/ADM_Spacing_declare.h"

c we're not taking any derivatives here, but if we were,
c we would also need the following line:
#include "EinsteinBase/ADMMacros/src/macro/ADM_Spacing.h"

<CUT>

c     Add the shift term: N = B^i D_i(trK).
      if ((maxshift).and.(shift_state.eq.1)) then
         do k=1,nz
            do j=1,ny
               do i=1,nx
#include "EinsteinBase/ADMMacros/src/macro/TRK_guts.h"
                  K_temp(i,j,k) = TRK_TRK
                end do
             end do
         end do
#include "EinsteinBase/ADMMacros/src/macro/TRK_undefine.h"
\end{verbatim}

\subsection{C}

This function computes the curved-space Laplacian of a scalar field $\phi$,
$\del^i \del_i \phi
	= g^{ij} \partial_{ij} \phi - g^{ij} \Gamma^k_{ij} \partial_k \phi$,
assuming that the partial derivatives $\partial_{ij} \phi$ and $\partial_k \phi$
have already been computed:

\newpage	% we'd like the entire example to fit on a single page
\begingroup
\footnotesize
\begin{verbatim}
/*
 * This function computes the curved-space Laplacian of a scalar field,
 * $\del^i \del_i \phi
 *     = g^{ij} \partial_{ij} \phi - g^{ij} \Gamma^k_{ij} \partial_k \phi$
 * at the interior grid points only; it doesn't do anything at all on the
 * boundaries.
 *
 * This function uses the following Cactus grid functions:
 *    input:   dx_phi, dy_phi, dz_phi       # 1st derivatives of phi
 *             dxx_phi, dxy_phi, dxz_phi,   # 2nd derivatives of phi
 *                      dyy_phi, dyz_phi,
 *                               dzz_phi
 *    output:  Laplacian_phi
 */
void compute_Laplacian(CCTK_ARGUMENTS)
{
DECLARE_CCTK_ARGUMENTS
int i,j,k;

/* contracted Christoffel symbols $\Gamma^k = g^{ij} \Gamma^k_{ij}$ */
CCTK_REAL Gamma_u_x, Gamma_u_y, Gamma_u_z;

/* grid-function strides for ADMMacros */
const int di = 1;
const int dj = cctk_ash[0];
const int dk = cctk_ash[0]*cctk_ash[1];

/* declare the ADMMacros variables for $g^{ij}$ and $\Gamma^k_{ij}$ */
#include "EinsteinBase/ADMMacros/src/macro/UPPERMET_declare.h"
#include "EinsteinBase/ADMMacros/src/macro/CHR2_declare.h"

    for (k = 1 ; k < cctk_lsh[2]-1 ; ++k)
    {
    for (j = 1 ; j < cctk_lsh[1]-1 ; ++j)
    {
    for (i = 1 ; i < cctk_lsh[0]-1 ; ++i)
    {
    const int ijk = CCTK_GFINDEX3D(cctkGH,i,j,k);   /* grid-function subscripting index for ADMMacros */
                                                    /* (must be assigned inside the i,j,k loops) */

    /* compute the ADMMacros $g^{ij}$ and $\Gamma^k_{ij}$ variables at the (i,j,k) grid point */
    #include "EinsteinBase/ADMMacros/src/macro/UPPERMET_guts.h"
    #include "EinsteinBase/ADMMacros/src/macro/CHR2_guts.h"

    /* compute the contracted Christoffel symbols $\Gamma^k = g^{ij} \Gamma^k_{ij}$ */
    Gamma_u_x =
      UPPERMET_UXX*CHR2_XXX + 2.0*UPPERMET_UXY*CHR2_XXY + 2.0*UPPERMET_UXZ*CHR2_XXZ
                            +     UPPERMET_UYY*CHR2_XYY + 2.0*UPPERMET_UYZ*CHR2_XYZ
                                                        +     UPPERMET_UZZ*CHR2_XZZ;
    Gamma_u_y =
      UPPERMET_UXX*CHR2_YXX + 2.0*UPPERMET_UXY*CHR2_YXY + 2.0*UPPERMET_UXZ*CHR2_YXZ
                            +     UPPERMET_UYY*CHR2_YYY + 2.0*UPPERMET_UYZ*CHR2_YYZ
                                                        +     UPPERMET_UZZ*CHR2_YZZ;
    Gamma_u_z =
      UPPERMET_UXX*CHR2_ZXX + 2.0*UPPERMET_UXY*CHR2_ZXY + 2.0*UPPERMET_UXZ*CHR2_ZXZ
                            +     UPPERMET_UYY*CHR2_ZYY + 2.0*UPPERMET_UYZ*CHR2_ZYZ
                                                        +     UPPERMET_UZZ*CHR2_ZZZ;

    /* compute the Laplacian */
    Laplacian_phi[ijk] =
      UPPERMET_UXX*dxx_phi[ijk] + 2.0*UPPERMET_UXY*dxy_phi[ijk] + 2.0*UPPERMET_UXZ*dxz_phi[ijk]
                                +     UPPERMET_UYY*dyy_phi[ijk] + 2.0*UPPERMET_UYZ*dyz_phi[ijk]
                                                                +     UPPERMET_UZZ*dzz_phi[ijk]
      -  Gamma_u_x*dx_phi[ijk]  -        Gamma_u_y*dy_phi[ijk]  -        Gamma_u_z*dz_phi[ijk];
    }
    }
    }

#include "EinsteinBase/ADMMacros/src/macro/UPPERMET_undefine.h"
#include "EinsteinBase/ADMMacros/src/macro/CHR2_undefine.h"
}
\end{verbatim}
\endgroup


\section{Macros}
\label{admmacros:macros}
Macros exist for the following quantities

\begin{tabular}{p{5cm}p{5cm}p{5cm}}
\textbf{Calculates} & \textbf{Macro Name} & \textbf{Sets variables} \\
All first spatial derivatives of lapse, $\alpha_{,i}$: & DA & DA\_DXDA, DA\_DYDA, DA\_DZDA\\
All second spatial derivatives of lapse, $\alpha_{,ij}$: & DDA & DDA\_DXXDA, DDA\_DXYDA, DDA\_DXZDA, DDA\_DYYDA, DDA\_DYZDA, DDA\_DZZDA\\
All second covariant spatial derivatives of lapse, $\alpha_{;ij}$: & CDCDA &\\
All first spatial derivatives of shift, $\beta^{i}_{\;\;j}$: & DB &\\
All first covariant derivatives of the extrinsic curvature, $K_{ij;kl}$ & CDK &\\
First covariant derivatives of the extrinsic curvature, $K_{ij;x}$, $K_{ij;y}$, $K_{ij;z}$ & CDXCDK, CDYCDK, CDZCDK &\\
Determinant of 3-metric: & DETG &\\
Upper 3-metric, $g{ij}$:& UPPERMET &\\
Trace of extrinsic curvature $trK$: & TRK &\\
Trace of stress energy tensor: & TRT &\\
Hamiltonian constraint: & HAMADM & \\
Partial derivatives of extrinsic curvature, $K_{ij,x}$, $K_{ij,y}$, $K_{ij,z}$: & DXDK, DYDK, DZDK &\\
First partial derivatives of 3-metric, $g_{ij,x}$, $g_{ij,y}$, $g_{ij,z}$: & DXDG, DYDG, DZDG & \\
All first partial derivatives of 3-metric, $g_{ij,k}$: & DG &\\
First covariant derivatives of 3-metric, $g_{ij;x}$, $g_{ij;y}$, $g_{ij;z}$: & DXDCG, DYDCG, DZDCG &\\
Second partial derivatives of 3-metric, $g_{ij,xx}$, $g_{ij,xy}$, $g_{ij,xz}$: & DXXDG, DXYDG, DXZDG, DYYDG, DYZDG, DZZDG& \\
All second partial derivative of 3-metric, $g_{ij,lm}$ & DDG &\\
Ricci tensor $R_{ij}$: & RICCI &\\
Trace of Ricci tensor $R$: & TRRICCI &\\
Christoffel symbols of first kind: $\Gamma_{cab}$ & CHR1&\\
Christoffel symbols of second kind $\Gamma^{c}_{\;\;ab}$: & CHR2& \\
Momentum constraints & MOMX, MOMY, MOMZ&\\
Source term in evolution equation for conformal metric, $\tilde{g}_{ij,t}$: & DCGDT &\\

\end{tabular}


\section{Definitions}

\begin{equation}
\Gamma_{cab} = \frac{1}{2}\left(g_{ac,b} + g_{bc,a} - g_{ab,c}\right)
\end{equation}

\begin{equation}
\Gamma^{c}_{\;\;ab} = g^{cd}\Gamma_{dab} = \frac{1}{2} g^{cd} \left(g_{ad,b} + g_{bd,a} - g_{ab,d}\right)
\end{equation}

% Do not delete next line
% END CACTUS THORNGUIDE


\section{Parameters} 


\parskip = 0pt

\setlength{\tableWidth}{160mm}

\setlength{\paraWidth}{\tableWidth}
\setlength{\descWidth}{\tableWidth}
\settowidth{\maxVarWidth}{spatial\_order}

\addtolength{\paraWidth}{-\maxVarWidth}
\addtolength{\paraWidth}{-\columnsep}
\addtolength{\paraWidth}{-\columnsep}
\addtolength{\paraWidth}{-\columnsep}

\addtolength{\descWidth}{-\columnsep}
\addtolength{\descWidth}{-\columnsep}
\addtolength{\descWidth}{-\columnsep}
\noindent \begin{tabular*}{\tableWidth}{|c|l@{\extracolsep{\fill}}r|}
\hline
\multicolumn{1}{|p{\maxVarWidth}}{spatial\_order} & {\bf Scope:} restricted & INT \\\hline
\multicolumn{3}{|p{\descWidth}|}{{\bf Description:}   {\em Order of spatial differencing}} \\
\hline{\bf Range} & &  {\bf Default:} 2 \\\multicolumn{1}{|p{\maxVarWidth}|}{\centering 2} & \multicolumn{2}{p{\paraWidth}|}{2nd order finite differencing} \\\multicolumn{1}{|p{\maxVarWidth}|}{\centering 4} & \multicolumn{2}{p{\paraWidth}|}{4th order finite differencing} \\\hline
\end{tabular*}

\vspace{0.5cm}\parskip = 10pt 

\section{Interfaces} 


\parskip = 0pt

\vspace{3mm} \subsection*{General}

\noindent {\bf Implements}: 

admmacros
\vspace{2mm}
\subsection*{Grid Variables}
\vspace{5mm}\subsubsection{PUBLIC GROUPS}

\vspace{5mm}

\begin{tabular*}{150mm}{|c|c@{\extracolsep{\fill}}|rl|} \hline 
~ {\bf Group Names} ~ & ~ {\bf Variable Names} ~  &{\bf Details} ~ & ~\\ 
\hline 
local\_spatial\_order & local\_spatial\_order & compact & 0 \\ 
 &  & description & Can be set pointwise to change the finite differencing order \\ 
 &  & dimensions & 0 \\ 
 &  & distribution & CONSTANT \\ 
 &  & group type & SCALAR \\ 
 &  & timelevels & 1 \\ 
 &  & variable type & INT \\ 
\hline 
\end{tabular*} 



\vspace{5mm}

\noindent {\bf Uses header}: 

CalcTmunu\_temps.inc

CalcTmunu.inc
\vspace{2mm}\parskip = 10pt 

\section{Schedule} 


\parskip = 0pt


\noindent This section lists all the variables which are assigned storage by thorn EinsteinBase/ADMMacros.  Storage can either last for the duration of the run ({\bf Always} means that if this thorn is activated storage will be assigned, {\bf Conditional} means that if this thorn is activated storage will be assigned for the duration of the run if some condition is met), or can be turned on for the duration of a schedule function.


\subsection*{Storage}

\hspace{5mm}

 \begin{tabular*}{160mm}{ll} 

{\bf Always:}&  ~ \\ 
 local\_spatial\_order & ~\\ 
~ & ~\\ 
\end{tabular*} 


\subsection*{Scheduled Functions}
\vspace{5mm}

\noindent {\bf CCTK\_BASEGRID} 

\hspace{5mm} admmacros\_setlocalspatialorder 

\hspace{5mm}{\it initialize the local\_spatial\_order } 


\hspace{5mm}

 \begin{tabular*}{160mm}{cll} 
~ & Language:  & c \\ 
~ & Type:  & function \\ 
~ & Writes:  & admmacros::local\_spatial\_order(everywhere) \\ 
\end{tabular*} 



\vspace{5mm}\parskip = 10pt 
\end{document}
