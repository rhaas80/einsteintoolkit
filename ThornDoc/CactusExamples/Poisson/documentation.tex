% *======================================================================*
%  Cactus Thorn template for ThornGuide documentation
%  Author: Ian Kelley
%  Date: Sun Jun 02, 2002
%  $Header$
%
%  Thorn documentation in the latex file doc/documentation.tex
%  will be included in ThornGuides built with the Cactus make system.
%  The scripts employed by the make system automatically include
%  pages about variables, parameters and scheduling parsed from the
%  relevant thorn CCL files.
%
%  This template contains guidelines which help to assure that your
%  documentation will be correctly added to ThornGuides. More
%  information is available in the Cactus UsersGuide.
%
%  Guidelines:
%   - Do not change anything before the line
%       % START CACTUS THORNGUIDE",
%     except for filling in the title, author, date, etc. fields.
%        - Each of these fields should only be on ONE line.
%        - Author names should be separated with a \\ or a comma.
%   - You can define your own macros, but they must appear after
%     the START CACTUS THORNGUIDE line, and must not redefine standard
%     latex commands.
%   - To avoid name clashes with other thorns, 'labels', 'citations',
%     'references', and 'image' names should conform to the following
%     convention:
%       ARRANGEMENT_THORN_LABEL
%     For example, an image wave.eps in the arrangement CactusWave and
%     thorn WaveToyC should be renamed to CactusWave_WaveToyC_wave.eps
%   - Graphics should only be included using the graphicx package.
%     More specifically, with the "\includegraphics" command.  Do
%     not specify any graphic file extensions in your .tex file. This
%     will allow us to create a PDF version of the ThornGuide
%     via pdflatex.
%   - References should be included with the latex "\bibitem" command.
%   - Use \begin{abstract}...\end{abstract} instead of \abstract{...}
%   - Do not use \appendix, instead include any appendices you need as
%     standard sections.
%   - For the benefit of our Perl scripts, and for future extensions,
%     please use simple latex.
%
% *======================================================================*
%
% Example of including a graphic image:
%    \begin{figure}[ht]
% 	\begin{center}
%    	   \includegraphics[width=6cm]{/home/runner/work/einsteintoolkit/einsteintoolkit/arrangements/CactusExamples/Poisson/doc/MyArrangement_MyThorn_MyFigure}
% 	\end{center}
% 	\caption{Illustration of this and that}
% 	\label{MyArrangement_MyThorn_MyLabel}
%    \end{figure}
%
% Example of using a label:
%   \label{MyArrangement_MyThorn_MyLabel}
%
% Example of a citation:
%    \cite{MyArrangement_MyThorn_Author99}
%
% Example of including a reference
%   \bibitem{MyArrangement_MyThorn_Author99}
%   {J. Author, {\em The Title of the Book, Journal, or periodical}, 1 (1999),
%   1--16. {\tt http://www.nowhere.com/}}
%
% *======================================================================*

% If you are using CVS use this line to give version information
% $Header$

\documentclass{article}

% Use the Cactus ThornGuide style file
% (Automatically used from Cactus distribution, if you have a
%  thorn without the Cactus Flesh download this from the Cactus
%  homepage at www.cactuscode.org)
\usepackage{../../../../../doc/latex/cactus}

\newlength{\tableWidth} \newlength{\maxVarWidth} \newlength{\paraWidth} \newlength{\descWidth} \begin{document}

% The author of the documentation
\author{Erik Schnetter \textless schnetter@gmail.com\textgreater}

% The title of the document (not necessarily the name of the Thorn)
\title{Poisson}

% the date your document was last changed, if your document is in CVS,
% please use:
%    \date{$ $Date$ $}
\date{November 25, 2014}

\maketitle

% Do not delete next line
% START CACTUS THORNGUIDE

% Add all definitions used in this documentation here
%   \def\mydef etc

% Add an abstract for this thorn's documentation
\begin{abstract}
  This is an example thorn describing how to use the TATPETSc
  interface to PETSc. It solves the Poisson equation for a spherical
  charge distribution on a uniform grid.
\end{abstract}

% The following sections are suggestive only.
% Remove them or add your own.

\section{Introduction}

PETSc is a well-known library for solving elliptic equations. TATPETSc
is a Cactus thorn that provides a wrapper for calling PETSc to solve
elliptic equations on uniform grids. (TATPETSc currently supports
neither mesh refinement nor multi-block systems.) TATPETSc can solve
both linear and non-linear systems.

\section{Physical System}

Here we solve the Poisson equation
\begin{eqnarray}
  \Delta\Phi(x) &=& \rho(x)
\end{eqnarray}
where the right hand side $\rho$ is given by
\begin{eqnarray}
  \rho(r) & = & \left\{
  \begin{array}{ll}
    Q/V & r\le R \\
    0 & r>R
  \end{array}
  \right.
\end{eqnarray}
for the charge $Q$ and the radius $R$, with $V=4\pi R^3/3$. We use
Dirichlet boundary conditions $\Phi(x)=0$.

\section{Numerical Implementation}

PETSc supports a large number of options to choose solvers. Here we
use PETSc's default settings.

\section{Using This Thorn}

In the example parameter file, we set the parameter
\texttt{TATPETSc::options} to select the following PETSc options:
\begin{itemize}
\item \verb+-snes_atol 1e-8+: set absolute tolerance for residual
\item \verb+-snes_stol 1e-8+: set relative tolerance for residual
\item \verb+-snes_monitor+: output progress information at each
  iteration of the non-linear solver
\item \verb+-ksp_monitor+: output progress information at each
  iteration of the linear (Krylov subspace) solver
\end{itemize}

\subsection{Examples}

The solution (the potential $\Phi(x)$) is stored in the grid function
\texttt{potential}, the residual (a measure for the error) in the grid
function \texttt{residual}.

% Do not delete next line
% END CACTUS THORNGUIDE



\section{Parameters} 


\parskip = 0pt

\setlength{\tableWidth}{160mm}

\setlength{\paraWidth}{\tableWidth}
\setlength{\descWidth}{\tableWidth}
\settowidth{\maxVarWidth}{options}

\addtolength{\paraWidth}{-\maxVarWidth}
\addtolength{\paraWidth}{-\columnsep}
\addtolength{\paraWidth}{-\columnsep}
\addtolength{\paraWidth}{-\columnsep}

\addtolength{\descWidth}{-\columnsep}
\addtolength{\descWidth}{-\columnsep}
\addtolength{\descWidth}{-\columnsep}
\noindent \begin{tabular*}{\tableWidth}{|c|l@{\extracolsep{\fill}}r|}
\hline
\multicolumn{1}{|p{\maxVarWidth}}{charge} & {\bf Scope:} private & REAL \\\hline
\multicolumn{3}{|p{\descWidth}|}{{\bf Description:}   {\em Charge of uniformly charged sphere}} \\
\hline{\bf Range} & &  {\bf Default:} 1.0 \\\multicolumn{1}{|p{\maxVarWidth}|}{\centering *:*} & \multicolumn{2}{p{\paraWidth}|}{} \\\hline
\end{tabular*}

\vspace{0.5cm}\noindent \begin{tabular*}{\tableWidth}{|c|l@{\extracolsep{\fill}}r|}
\hline
\multicolumn{1}{|p{\maxVarWidth}}{options} & {\bf Scope:} private & STRING \\\hline
\multicolumn{3}{|p{\descWidth}|}{{\bf Description:}   {\em Options for the solver}} \\
\hline{\bf Range} & &  {\bf Default:} (none) \\\multicolumn{1}{|p{\maxVarWidth}|}{\centering .*} & \multicolumn{2}{p{\paraWidth}|}{no restriction} \\\hline
\end{tabular*}

\vspace{0.5cm}\noindent \begin{tabular*}{\tableWidth}{|c|l@{\extracolsep{\fill}}r|}
\hline
\multicolumn{1}{|p{\maxVarWidth}}{radius} & {\bf Scope:} private & REAL \\\hline
\multicolumn{3}{|p{\descWidth}|}{{\bf Description:}   {\em Radius of uniformly charged sphere}} \\
\hline{\bf Range} & &  {\bf Default:} 1.0 \\\multicolumn{1}{|p{\maxVarWidth}|}{\centering 0:*} & \multicolumn{2}{p{\paraWidth}|}{} \\\hline
\end{tabular*}

\vspace{0.5cm}\noindent \begin{tabular*}{\tableWidth}{|c|l@{\extracolsep{\fill}}r|}
\hline
\multicolumn{1}{|p{\maxVarWidth}}{solver} & {\bf Scope:} private & STRING \\\hline
\multicolumn{3}{|p{\descWidth}|}{{\bf Description:}   {\em Name of TATelliptic solver that should be used}} \\
\hline{\bf Range} & &  {\bf Default:} TATJacobi \\\multicolumn{1}{|p{\maxVarWidth}|}{\centering .*} & \multicolumn{2}{p{\paraWidth}|}{must be an activated TATelliptic solver} \\\hline
\end{tabular*}

\vspace{0.5cm}\parskip = 10pt 

\section{Interfaces} 


\parskip = 0pt

\vspace{3mm} \subsection*{General}

\noindent {\bf Implements}: 

poisson
\vspace{2mm}

\noindent {\bf Inherits}: 

boundary

grid
\vspace{2mm}
\subsection*{Grid Variables}
\vspace{5mm}\subsubsection{PRIVATE GROUPS}

\vspace{5mm}

\begin{tabular*}{150mm}{|c|c@{\extracolsep{\fill}}|rl|} \hline 
~ {\bf Group Names} ~ & ~ {\bf Variable Names} ~  &{\bf Details} ~ & ~\\ 
\hline 
potential &  & compact & 0 \\ 
 & phi & description & Potential for elliptic equation \\ 
 &  & dimensions & 3 \\ 
 &  & distribution & DEFAULT \\ 
 &  & group type & GF \\ 
 &  & timelevels & 1 \\ 
 &  & variable type & REAL \\ 
\hline 
residual &  & compact & 0 \\ 
 & res & description & Residual for elliptic equation \\ 
 &  & dimensions & 3 \\ 
 &  & distribution & DEFAULT \\ 
 &  & group type & GF \\ 
 &  & timelevels & 1 \\ 
 &  & variable type & REAL \\ 
\hline 
\end{tabular*} 



\vspace{5mm}

\noindent {\bf Uses header}: 

carpet.h

TATelliptic.h
\vspace{2mm}\parskip = 10pt 

\section{Schedule} 


\parskip = 0pt


\noindent This section lists all the variables which are assigned storage by thorn CactusExamples/Poisson.  Storage can either last for the duration of the run ({\bf Always} means that if this thorn is activated storage will be assigned, {\bf Conditional} means that if this thorn is activated storage will be assigned for the duration of the run if some condition is met), or can be turned on for the duration of a schedule function.


\subsection*{Storage}

\hspace{5mm}

 \begin{tabular*}{160mm}{ll} 

{\bf Always:}&  ~ \\ 
 potential residual & ~\\ 
~ & ~\\ 
\end{tabular*} 


\subsection*{Scheduled Functions}
\vspace{5mm}

\noindent {\bf CCTK\_INITIAL} 

\hspace{5mm} poisson\_prepare 

\hspace{5mm}{\it set up initial guess for initial data } 


\hspace{5mm}

 \begin{tabular*}{160mm}{cll} 
~ & Language:  & c \\ 
~ & Type:  & function \\ 
~ & Writes:  & poisson::phi(everywhere) \\ 
\end{tabular*} 


\vspace{5mm}

\noindent {\bf CCTK\_POSTINITIAL} 

\hspace{5mm} poisson\_solve 

\hspace{5mm}{\it calculate uniform charge initial data } 


\hspace{5mm}

 \begin{tabular*}{160mm}{cll} 
~ & Language:  & c \\ 
~ & Reads:  & poisson::phi(everywhere) \\ 
~& ~ &grid::coordinates(interior)\\ 
~ & Type:  & function \\ 
~ & Writes:  & poisson::res(interior) \\ 
\end{tabular*} 


\vspace{5mm}

\noindent {\bf } 

\hspace{5mm} poisson\_boundaries 

\hspace{5mm}{\it apply boundary conditions to initial data } 


\hspace{5mm}

 \begin{tabular*}{160mm}{cll} 
~ & Type:  & group \\ 
\end{tabular*} 


\vspace{5mm}

\noindent {\bf Poisson\_boundaries} 

\hspace{5mm} poisson\_boundaries\_select 

\hspace{5mm}{\it select boundary conditions for initial data } 


\hspace{5mm}

 \begin{tabular*}{160mm}{cll} 
~ & Language:  & c \\ 
~ & Options:  & level \\ 
~ & Sync:  & potential \\ 
~ & Type:  & function \\ 
\end{tabular*} 


\vspace{5mm}

\noindent {\bf Poisson\_boundaries} 

\hspace{5mm} applybcs 

\hspace{5mm}{\it apply boundary conditions to initial data } 


\hspace{5mm}

 \begin{tabular*}{160mm}{cll} 
~ & After:  & poisson\_boundaries\_select \\ 
~ & Type:  & group \\ 
\end{tabular*} 


\subsection*{Aliased Functions}

\hspace{5mm}

 \begin{tabular*}{160mm}{ll} 

{\bf Alias Name:} ~~~~~~~ & {\bf Function Name:} \\ 
ApplyBCs & Poisson\_boundaries\_apply \\ 
\end{tabular*} 



\vspace{5mm}\parskip = 10pt 
\end{document}
